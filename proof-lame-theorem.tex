  (a)
Let $P(n)$ be the statement that if $a > b > 0$ are integers
such that the Euclidean algorithm computation of $\gcd(a,b)$
results in $n$ steps, then 
$a \geq F_{n+2}$ and
$b \geq F_{n+1}$.
We will prove that $P(n)$ holds for all $n \geq 1$ using
weak induction.

For the base case of $n = 1$,
since $b > 0$, we have $b \geq 1 = F_2$.
Since $a > b \geq 1$, we have $a \geq 2 = F_3$.
Hence the base case $P(1)$ holds.

Now assume that $P(n - 1)$ holds.
We want to show that $P(n)$ holds.
The first step in the Euclidean algorithm for $\gcd(a, b)$
gives us
\[
\gcd(a, b) = \gcd(b, r)
\]
where
\[
a = b q + r, \,\,\, 0 \leq r < b
\]
By induction hypothesis, since the number of steps in
the Euclidean algorithm computation of 
$\gcd(b, r)$ takes $n - 1$ steps, we have
\[
b \geq F_{(n - 1) + 2}, \,\,\, r \geq F_{(n - 1) + 1}
\]
i.e.,
\[
b \geq F_{n + 1}, \,\,\, r \geq F_{n}
\]
Hence
\begin{align*}
  a
  &= b q + r \\
  &\geq b + r & \textit{because $q = \floor{a/b} \geq 1$ since $a \geq b$} \\
  &\geq F_{n + 1} + F_n \\
  &= F_{n + 2}
\end{align*}
Altogether we have $a \geq F_{n + 2}$ and $b\geq F_{n + 1}$.
Hence $P(n)$ holds.

Therefore by weak induction, $P(n)$ holds for all $n \geq 1$.

(b)
Recall that
\[
F_n = \frac{1}{\sqrt{5}}
\left(
\left(\frac{1 + \sqrt{5}}{2}\right)^n
-
\left(\frac{1 - \sqrt{5}}{2}\right)^n
\right)
\]
Let $\phi = \frac{1 + \sqrt{5}}{2}$ (the golden ratio).
From the previous theorem, we have
\[
b \geq F_{n + 1} \tag{1}
\]
We now link $b$ to $n$ more directly.
We claim that $F_{n + 1} \geq \phi^{n - 1}$.
We will prove this by strong induction.
For $n = 0$, $F_1 = 1 = \phi^{n - 1}$.
Suppose $F_{k} \geq \phi^{k - 2}$ for $k < n + 1$.
Then, by inductive hypothesis
\[
F_{n + 2} = F_{n + 1} + F_{n} \geq \phi^{n-1} + \phi^{n-2}
\]
We claim that 
$\phi^{n-1} + \phi^{n-2} = \phi^n$.
This follows from $\phi^1 + \phi^0 = \phi^2$ which can be easily verified:
\begin{align*}
\phi^2
&= \left(\frac{1 + \sqrt{5}}{2}\right)^2
= \frac{1 + 2\sqrt{5} + 5}{4}
= \frac{3 + \sqrt{5}}{2}
= \frac{1 + \sqrt{5}}{2} + 1
= \phi + 1
\\
&
= \phi^1 + \phi^0
\end{align*}
Multiplying this equation by $\phi^{n-2}$ gives us
$\phi^n = \phi^{n-1} + \phi^{n-2}$.
Hence
\[
F_{n + 2} \geq \phi^{n-1} + \phi^{n-2} = \phi^n
\]
Continuing (1), we get
\begin{align*}
b &\geq F_{n+1} \geq \phi^{n-1} \\
\THEREFORE \log_{\phi} b &\geq n-1 \\
%\THEREFORE n - 1 &\leq \log_{\phi} b = \frac{\log_{10} b}{\log_{10} \phi} \\
\THEREFORE n - 1 &\leq \log_{\phi} b = \frac{\log_{10} b}{\log_{10} \phi}
= (\log_{10} b)  \times 4.7849... < 5 \log_{10}b\\
\THEREFORE n &<  5 \log_{10}b + 1 = 5 (\log_{10}b + 1) - 4 \tag{2}
\end{align*}
Let $\alpha$ be the fractional part of $\log_{10}b + 1$, i.e.,
$\log_{10}b + 1 = \floor{\log_{10}b + 1} + \alpha$ with $0 \leq \alpha < 1$.
Continuing (2),
\begin{align*}
n
&< 5 (\floor{\log_{10}b + 1} + \alpha) - 4 \\
&= 5 \floor{\log_{10}b + 1} + 5\alpha - 4 \\
\THEREFORE n 
&\leq 5 \floor{\log_{10}b + 1} + \floor{5\alpha - 4} \\
&\leq 5 \floor{\log_{10}b + 1}
\end{align*}
The fact that the number of digits in $b$ is 
$\floor{\log_{10}b + 1}$ is proven below.
\qed
