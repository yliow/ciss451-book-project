\section{Exponentiation: the squaring method}

See CISS358. Skip this section if you have taken CISS358.

Note that for RSA we need to compute powers. Extremely large
powers.
The obvious method: If $n > 0$,
\[
a^n = a^{n-1} \cdot a
\]
has a runtime of $O(n)$ (in fact $\Theta(n)$) of multiplications.
If the exponentiation is in mod $N$, then
you take mod $N$ after each multiplication to previous
the $a^k$ from becoming too huge, i.e.,
\[
a^k = a^{k-1} \cdot a \pmod{N}
\]

There's a much faster way to compute exponentiation.

First I will describe it using recursion:
\[
a^n =
\begin{cases}
  1 & n = 0 \\
  \left( a^{n/2} \right)^2 & n > 0 \text{ and $n$ is even} \\
  a \cdot \left( a^{(n - 1)/2} \right)^2 & n > 0 \text{ and $n$ is odd} \\
\end{cases}
\]
Here, $n \geq 0$.

\begin{eg}
Here's an example to compute $2^{27}$
\begin{enumerate}[nosep]
\item $2^{27} = 2 \cdot \left( 2^{13} \cdot 2^{13} \right) $
\item $2^{13} = 2 \cdot \left( 2^{6} \cdot 2^{6} \right) $
\item $2^6 = \left( 2^3 \cdot 2^3 \right)$
\item $2^3 = 2 \cdot \left( 2^1 \cdot 2^1 \right)$
\item $2^1 = 2 \cdot \left( 2^0 \cdot 2^0 \right) = 2 \cdot \left( 1 \cdot 1 \right)$
\end{enumerate}
\end{eg}

It's easy to write an algorithm implementing the above:
\begin{Verbatim}[frame=single,fontsize=\footnotesize]
ALGORITHM: power
INPUTS: a, n where n >= 0

if n == 0:
    return 1
else:
    if n is even: 
        x = power(a, n / 2)
        return x * x
    else:
        x = power(a, (n - 1) / 2)
        return a * x * x
\end{Verbatim}
The number of recursions is $O(\lg n)$ where the main
cost of each recursion step is either one or two multiplications.

This above recursion can be rewritten using a loop.
Suppose you want to compute $a^x$.
First you write $x$ as a binary number:
\[
x = (x_k\cdots x_0)_2 = x_k 2^k + \ldots + x_12^2 + x_02^0
\]
where each $x_i$ is $0$ or $1$.
As an example suppose we look at
\[
x = 27 = (11011)_2 =  1 \cdot 2^4 + 1 \cdot 2^3 + 0 \cdot 2^2 + 1 \cdot 2^1 + 1 \cdot 2^0
\]
Then
\[
a^{27} \equiv a^{1 \cdot 2^4 + 1 \cdot 2^3 + 0 \cdot 2^3 + 1 \cdot 2^1 + 1 \cdot 2^0} \equiv a^{2^4}a^{2^3}a^{2^1}a^{2^0}
\]
Notice that the computation of $a^{27}$ on the right
depends on $a^{2^0},a^{2^1},a^{2^2},...$
In general $a^{2^{i+1}} = (a^{2^i})^2$.
In the following $b$ runs through $a^{2^0},a^{2^1},a^{2^2},...$.
Notice that $a^{2^i}$ is included in $a^{27}$ if the $i$--th bit of $x$ is $1$.
Instead of pre-computing the bits of $x$, we can compute the $i$--th bit
iteratively as the exponentiation is computed.
In the following $p$ runs through the partial products of $a^{2^0},a^{2^1},a^{2^2},...$,
picking up $a^{2^i}$ if the $i$--th bit of $n$ is $1$.
In this case, $p$ and $b$ will run through the following values:
\begin{align*}
p = 1                                                         & \hspace{1cm} b = a = a^{2^0} \\
p = p \cdot b = a^{2^0}                                        & \hspace{1cm} b = b \cdot b = a^{2^1}\\
p = p \cdot b = a^{2^0} \cdot a^{2^1}                            & \hspace{1cm} b = b \cdot b = a^{2^2}\\
                                                              & \hspace{1cm} b = b \cdot b = a^{2^3} \\
p = p \cdot b = a^{2^0} \cdot a^{2^1} \cdot a^{2^3}               & \hspace{1cm} b = b \cdot b = a^{2^4} \\
p = p \cdot b = a^{2^0} \cdot a^{2^1} \cdot a^{2^3}  \cdot a^{2^4} & \hspace{1cm} b = b \cdot b = a^{2^5}
\end{align*}
Here's the algorithm:
\begin{Verbatim}[frame=single,fontsize=\footnotesize]
ALGORITHM: power
INPUTS: a, n where n >= 0
OUTPUT: a^n

p = 1
b = a

while n is not 0:
    bit = n % 2
    n = n / 2 (integer division)
    if bit == 1:
        p = p * b
    b = b * b

return p    
\end{Verbatim}

And if the exponentiation is done in mod $N$, then we just
mod by $N$ as frequently as we can:
\begin{Verbatim}[frame=single,fontsize=\footnotesize]
ALGORITHM: power-mod
INPUTS: a, n, N where n >= 0 and N is the modulus
OUTPUT: (a^n) % N

p = 1
b = a % N
while n is not 0:
    if n % 2 == 1:
        p = (p * b) % N
    n = n // 2 
    b = (b * b) % N
return p    
\end{Verbatim}

The number of iterations is the number of bits of $n$, i.e.,
the number of iterations is $\floor{\lg n + 1}$ ($\lg = \log_2$).
Each iteration in the worse case scenario involves
two mod $N$ multiplications.
Hence the runtime time is
\[
O((\lg n) \cdot M)
\]
where $M$ is the runtime for multiplication of two integers in $\Z/N$.
(In the case when we are performing exponentiation without mod $N$,
the length of the integer \verb!p! in the above increases.
In that case the factor $M$ for the runtime of multiplication increases.)


To include the case of negative exponent, when $a$ is a real number
\[
a^{-n} = (a^{-1})^n
\]
and for mod $N$, if $a$ is invertible,
\[
a^{-n} = (a^{-1})^n \pmod{N}
\]

\begin{ex}
  Leetcode 50.\\
  \url{https://leetcode.com/problems/powx-n/}\\
  Implement \verb!pow(x, n)!,
  which calculates \verb!x! raised to the power
  \verb!n!.
\end{ex}


\begin{ex}$^*$
  In the above, the computation of $a^x$ depends on writing $x$ is base 2.
  What if you write $x$ in base 3?
\end{ex}
