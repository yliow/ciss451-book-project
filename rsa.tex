\section{RSA}

Now for RSA $\ldots$

We only need to work with integers. Why? Because any message $M$ is
really just a sequence of bits and you can cut your sequences of
bits which can be viewed as an unsigned int, i.e, an integer $\geq 0$.
So we'll just think of our messages as integers.

So once again suppose Alice wants to send a secret to Bob. The
secret is an integer $x$.

Bob selects two distinct primes $p$ and $q$.
He computes $N = pq$ and selected two positive integers $e$ and $d$
such that
\[
 ed \equiv 1 \,\,\,(\operatorname{mod} \phi(N))
\]
In other words $e$ and $d$ and multiplicative inverses of each
other in $\Z/\phi(N)$.
Furthermore Bob publishes $N$ and $e$ publicly.
$e$ is called the \defone{encryption exponent}
and
$d$ is called the \defone{decryption exponent}.

We assume that the secret $x$ is less than $N$
because we'll be working in $\Z/N$.
(Again if $x$ as a bit sequence is too large for $N$,
then we cut $x$ up into smaller block of bits and send them
separately.)

Since $N$ and $e$ are public, Alice can download $N$ and $e$
and then compute
\[
E_{(N,e)}(x) = x^e \,\,\,(\operatorname{mod} N)
\]
I'll write $x^e \,\,\,(\operatorname{mod} N)$ for the 
least positive remainder of $x^e$ mod $N$.
She then sends $x^e \,\,\,(\operatorname{mod} N)$ to Bob.

When Bob received $x^e \,\,\,(\operatorname{mod} N)$, he computes
\[
D_{(N,d)}(x) = x^{ed} \,\,\,(\operatorname{mod} N)
\]
What we need to prove is that RSA works, i.e.,


\begin{thm}
  Let $p,q$ be primes and $N = pq$.
  If $ed \equiv 1 \,\,\,(\operatorname{mod} \phi(N))$,
  then
  \[
  (x^e)^d \equiv x \,\,\,(\operatorname{mod} N)
  \]
  for all integers $x$.
\end{thm}

\proof
Since $ed \equiv 1 \,\,\,(\operatorname{mod} \phi(N))$, we have
\[
 ed = k \phi(N) + 1
\]
for some integer $k$. Therefore
\[
x^{ed} = x^{k\phi(N) + 1} = (x^{\phi(N)})^k x
\]
We now consider several cases based on the possible values
of $\gcd(x, N)$.
Since $N = pq$, $\gcd(x, N)$ is $1$, $p$, $q$ or $pq$.
since the only possible divisors of $N=pq$ are $1,p,q,pq$.
The case of
$\gcd(x, N) = p$
and
$\gcd(x, N) = q$
are similar.

\textsc{Case 1: $\gcd(x, N) = 1$.}
TODO

\textsc{Case 2: $\gcd(x, N) = N$.}
TODO

\textsc{Case 3: $\gcd(x, N) = p$.}
\qed

\begin{prop}
  Prove the following:
  \begin{enumerate}[nosep,label=\textnormal{(\alph*)}]
  \item
    If $p$ and $q$ are distinct primes such that $p \mid a$ and $q \mid a$,
    then $pq \mid a$.
  \item 
    If $x \mid a$ and $y \mid a$ and $\gcd(x,y) = 1$, then $xy \mid a$.
  \item
    If $x \mid a$ and $y \mid a$, then $(xy/\gcd(x,y)) \mid a$.
  \end{enumerate}
  (b) is a generalization of (a)
  and (c) is a generalization of (b).
  \qed
\end{prop}
\proof
(a) TODO

(b)
Since $x \mid a$ and $y \mid a$, we have
$xk = a = y\ell$.
Hence $y \mid xk$.
Since $\gcd(x,y) = 1$,
by Extended Euclidean Algorithm, there are integer $A, B$ such that
\[
Ax + By = 1
\]
Hence
$Axk + Byk = k$.
Since $y$ divides $Byk$ and $Axk$, by linearity of divisibility,
$y$ divides $k$.
Hence $ym = k$.
Therefore $a = xk = xym$, i.e., $xy \mid a$.

(c)
Since $y \mid a$, $y/\gcd(x,y) \mid a$.
Since $\gcd(x, y/\gcd(x,y)) = 1$, by (b), $xy/\gcd(x,y) \mid a$.
\qed

Note that RSA (and all public key ciphers) are not
meant for encrypting messages like strings (emails, sales receipts,
image files, etc.)
This is an unfortunate thing that's done in many cryptography textbooks. 
They are actually used to encrypt/decrypt keys for private ciphers
(3DES, AES, etc.)
In particular the RSA standard
(called PKCS -- you can easily find lots of webpages on PKCS)
does not include
specification on how to break up long messages before encryption and how to
reassembly them after decryption.
The reason for using RSA (and other public key cipher) is used to
setup keys for private key cipher (example: AES) is because
AES is much faster than RSA.

OK, let's summarize everything.
In the following, I'll write $x \pmod{N}$ to be the remainder when $x$
is divided by $N$.
There are three steps: Bob has to
generate keys, Alice has to encrypt, and Bob has to decrypt.
\begin{enumerate}[nosep]
 \item Key Generation:
  \begin{enumerate}
   \item Bob selects distinct primes $p$ and $q$.
   \item Bob computes $N = pq$.
   \item Bob computes $\phi(N) = (p-1)(q-1)$.
   \item Bob selects $e$ such that $0 < e < \phi(N)$,
   $\gcd(e, \phi(n)) = 1$.
   \item Bob computes $d$ such that
     $ed \equiv 1 \pmod{\phi(N)}$.
 \item Bob publishes $(N,e)$ (the public key) but keeps $(N,d)$
   (the private key) to himself.
  \end{enumerate}
 \item Encryption:
   Alice obtains the publicly available $(N,e)$ (the public key) and
   computes
  \[
   E_{(N,e)}(x) = x^e \,\,\,(\operatorname{mod} N)
  \]
  and sends it to Bob.
 \item Decryption: Bob uses $(N,d)$ (the private key) to compute
 \[
  D_{(N,d)}( x^e \,\,\,(\operatorname{mod} N) ) = x^{ed} \,\,\,(\operatorname{mod} N)
 \]
\end{enumerate}

Note that the key is made up of
\begin{enumerate}[nosep]
  \li a \defone{public key} $(N,e)$ and
  \li a \defone{private key} $(N,d)$
\end{enumerate}
$(N,e)$ is revealed to the public.
$(N,d)$ is kept private.
In general

\begin{defn}
  A \defone{public cipher} is made up of the encryption
  and decryption functions
  $E_{\text{pubkey}}, D_{\text{privkey}}$ which depends on
  the key
  $k = (\text{pubkey}, \text{privkey})$
  which is a 2-tuple made up of
  the public key and private key.
  Such a cipher is also called an
  \defone{asymmetric key cipher}.
\end{defn}

Recall that in the case of private (or symmetric) key cipher
the encryption and decryption keys are the same. 

\begin{ex}
  Eve saw Alice sent the ciphertext $230539333248$ to Bob.
  Eve checks Bob's website and found out that his public RSA key is
  $(N, e) = (100000016300000148701, 7)$.
  Help Eve compute the plaintext.
  In fact, compute the private key $(N, d)$.
  You only need to compute $d$ since $N$ is known.
  How much time did you use?
  (Hint: Why is factoring $N$ crucial?)
  \qed
\end{ex}
