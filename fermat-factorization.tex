\section{Fermat factorization}

Suppose you want to factorize $n$ and you know that
$n$ is a difference of two squares:
\[
n = x^2 - y^2
\]
then right away you know
\[
n = (x + y)(x - y)
\]
Of course if $x - y$ is 1, the factorization is not
helpful at all.

To achieve the goal of writing $n$ as a difference of squares,
you can do the following:
\begin{tightlist}
  \li Check if $1^2 - n$ is a square.
  \li Check if $2^2 - n$ is a square.
  \li Check if $3^2 - n$ is a square.
  \li Etc.
\end{tightlist}
Why?
Because if $x^2 - n$ is a square, say $y^2$, then
$x^2 - n = y^2$ gives us
\[
n = x^2 - y^2
\]

Note that if $n = ab$ is odd (which is the case for an RSA modulus),
then writing $n$ as a difference of two squares since
\[
  n
  =
  \left(\frac{a + b}{2}\right)^2
  -
  \left(\frac{a - b}{2}\right)^2
\]
If $n$ is odd, then $a$ and $b$ are odd and
therefore $a + b$ and $a - b$ are both even.
Hence every odd $n$ can be expressed as the
difference of two squares.
What if $n$ is even?
Then you can remove $2^k$ from $n$
where $k$ is maximal to get
$n = 2^k n'$ and proceed with the above
idea applied to $n'$ if $n' > 1$.
In the following, I will assume $n$ is odd.

Now we do not want the factorization $n = n \times 1$.
Since we are aiming for $n = (x-y)(x+y)$,
one way is to start $x$ at around $\sqrt{n}$.
Then $y$ will be close to $0$ and therefore $x+y$ and $x-y$ won't be close to 1.

Another thing to note is that if we start $x$ at $\sqrt{n}$,
if $n$ is squarefree (which is the case for RSA),
we will have $x = \floor{\sqrt{n}} < \sqrt{n}$,
which means $x^2 - n$ is negative and therefore cannot be $y^2$.
In this case, we might as well start with
$x = \floor{\sqrt{n}} + 1$.

To find $x,y$ such that $n = x^2 - y^2$,
we try different values of $x$ and then check if
\[
  x^2 - n
\]
is a square:
\[
  x^2 - n = y^2
\]
Note that if $n$ is itself a square (this won't happen for the RSA case),
then the above will end with $y = 0$.

\begin{console}[fontsize=\footnotesize]
ALGORITHM: Fermat-Factorization
INPUT: n -- an odd integer

let x = floor(sqrt(n)) + 1
while x * x - n is not a square:
    x = x + 1

y = sqrt(x * x - n)
return (x - y, x + y)    
\end{console}

Here's an example execution of my code
for $n = 11 \cdot 11 \cdot 37$:
\VerbatimInput[frame=single,fontsize=\footnotesize]{fermat_factorization_example.tex}

Note that Fermat factorization need not be faster than trial division,
but it is specifically crafted to handle the product of two primes which are
\lq\lq close".

I'll give you one optimization:

\begin{ex}
Here's a slight optimization:
You need to compute $x^2$ and then $(x + 1)^2$, etc.
Clearly $(x + 1)^2 = x^2 + 2x + 1$.
So if you have $x$ and $x^2$, what's a faster way to compute $(x + 1)^2$?
Therefore if you have $x$ and $x^2 - n$, how would you compute
$(x+1)^2 - n$?
\end{ex}

\begin{ex}
  We also need to compute $\floor{\sqrt{n}}$, i.e., integer square root
  of $n$.
  How many algorithms can you come up with?
\end{ex}

\begin{ex}
  Implement Fermat factorization and try to optimize it as much as you can.
  (There are some questions below on optimization.)
  Test for many values of $n$.
  Make a plot.
  What is the runtime of Fermat factorization algorithm based on your experiment?
  You don't have to completely factorize $n$.
  Finding a factor is good enough.
  Can you get roughly $O(n^{1/2})$? $O(n^{1/3})$? $O(n^{1/4})$?
\end{ex}

\begin{ex}
  Implement brute force trial-by-division until $\sqrt{n}$ for $n$
  to find a factor of $n$.
  Compare the runtimes for this algorithm
  against your implementation of Fermat factorization above.
  Which one is faster?
  Remember that brute force performs $\floor{\sqrt{n}}$ mods in the worse case scenario.
\end{ex}

\begin{ex}
  If $n$ is an integer, how do you handle the check \lq\lq $n$ is a square"?
  Can you think of several ways to do that?
  What are their runtimes?
\end{ex}


\begin{ex}
  Suppose $z^2$ is an integer square that is the largest integer square that is $\leq$ to $x^2 - n$.
  For instance if $x^2 - n$ is $26$, then $z^2$ is $25$ (i.e., $z$ is 5).
  In other words $z^2$ is the square that is closest to $x^2 - n$ on the left.
  Note $z^2$ will be $x^2 - n$ if $x^2 - n$ is a square.
  How would you compute the the square closest to $(x + 1)^2 - n$ on the left?
  Is it $(z + 1)^2$? $(z + 2)^2$?
  If you can prove, for instance, that it's $(z + 2)^2$? then we can save
  on taking square root, correct?
  You can assume that you have the values of $x$ and $z$ and $n$.
\end{ex}

\begin{ex}
  Base on $x^2 - n$ and $(x + 1)^2 - n$, is there a way to guestimate
  $k$ such that
  $(x + k)^2 - n$ is a square or closed to a square?
%  \begin{comment}
%    Suppose
%    \begin{tightlist}
%    \item $y^2 \leq x^2 - n < (y + 1)^2$ (with LHS in fact $<$)
%    \item $z^2 \leq (x + 1)^2 - n < (z + 1)^2$
%    \end{tightlist}
%  \end{comment}
%  \begin{comment}
%    Suppose $y^2 < x^2 - n < (y + 1)^2$, i.e., $x^2 - n$ is not a square.
%    We have $x$ and $y$ and therefore know $x - y$.
%    Suppose $(x + 1)^2 - n$ is a square.
%    Will it be $(y + 1)^2$?
%    According to the test case earlier, it's not $y + 1$.
%    Suppose it is:
%    \[
%    (y+1)^2 = (x+1)^2 - n
%    \]
%    \[
%    y^2 + 2y + 1 = x^2 + 2x + 1 - n 
%    \]
%    \[
%    y^2 + 2y = x^2 + 2x - n 
%    \]
%    \[
%    y^2  = x^2 - n + 2(x - y) 
%    \]
%    We have
%    \[
%    y^2 < x^2 - n
%    \]
%    and we have $x$ and $y$.
%    We compute $2(x - y)$ quickly and check if $(x^2 - n) + 2(x - y)$ is $y^2$.
%    But $x > y$. So $2(x - y) > 0$.
%    So the $(y + 1)^2$ won't work.
%  \end{comment}
\end{ex}


\begin{comment}
  Square check:

  n^2 \equiv 0,1 (4)               --> 50%                          00                   01
  n^2 \equiv 0,1,4 (8)             --> 3/8 = 37.5%               000   100              001
  n^2 \equiv 0,1,4,9 (16)          --> 4/16 = 25%               0000  0100         0001     1001
  n^2 \equiv 0,1,4,9,16,17,25 (32) --> 7/32 = 21.8%            00000 00100        *0001    *1001

  Therefore is 1st 5 bits not 00000 or 00100 or *0001 or *1001, n is not square
  
  x^2 - n ... find first z^2     ---> z^2 <= x^2 - n
  (x+1)^2-n = x^2 - n + 2*x + 1, (z+1)^2 = z^2 + 2z + 1 ---> (z+1)^2 <= (x+1)^2 - n

  x^2 - n increase by 2x + 1
  z^2     increase by 2z + 1
  DIFF = 2(z - x)
  
  DELTA = (x^2 - n) - z^2.


  x^2 -> (x+1)^2 = x^2 + 2x + 1
  For k steps:
  (x+k)^2 = x^2 + 2kx + 1
  Find best k.
  
\end{comment}






\begin{ex}
  Suppose $n$ is an odd prime. 
  What will happen during the execution of Fermat's factorization algorithm with $n$ as input?
  Would the algorithm terminate? (If not, then it's not really an algorithm!)
  Can Fermat factorization algorithm be converted to a primality testing algorithm?
  Would it be possible? Would it be a good idea?
\end{ex}

\begin{comment}
  It would terminate when $x = (p + 1)/2$ and $y = (p - 1)/2$
  so that $x - y = 1$, $x + y = p$, i.e., we get the factorization $n = 1 \times n$.

  In this case $x$ starts at around $\sqrt{p} + 1$ and reaches $(p + 1)/2$,
  roughly from $\sqrt{p}$ to $p/2$.
  For instance if $p$ is about $10000$, then $p$ runs from $100$ to $5000$, i.e.,
  $4900$ integers.
  Brute force test for primeness runs from $2$ to $\sqrt{p}$ which in this case if $2$ to $100$.
  Therefore Fermat Factorization would be a very bad prime testing algorithm.
\end{comment}





% http://www.ams.org/journals/mcom/1999-68-228/S0025-5718-99-01133-3/home.html

\begin{ex}
  Write a parallel program for the Fermat factorization algorithm.
\end{ex}


\begin{ex}
  What if you test for the condition $n = x^3 - y^3$?
  Is there are version of
  factorization using difference of cubes?
  %We have
  %\[
  %  x^3 - y^3 = (x - y)(x^2 + xy + y^2)
  %\]
  %Suppose $n = abc$.
  %\[
  %  \frac{}{}
  %\]
\end{ex}
