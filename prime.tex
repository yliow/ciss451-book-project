\section{Primes}

\begin{defn}
A prime $p$ is a positive integer greater than 1 that is
divisible by only 1 and itself.
In other words $p \in \N$ is a
\defterm{prime}\index{prime}\tinysidebar{prime}
if $p > 1$ and 
if $d \mid p$, then $d = 1$ or $d = p$.
\end{defn}

Examples of primes are $2, 3, 5, 7, 11, 13, 17, 19, ...$.

Integers least zero can be divided into the following types:
\begin{enumerate}[nosep]
  \li $0$ -- the zero element
  \li $1$ -- the unit element (i.e. the only invertible element $\geq 0$)
  \li primes -- $2, 3, 5, 7, 11, ...$
  \li \defone{composites} -- integers $> 1$ which are not primes 
\end{enumerate}

(
Instead of primes of $\N \cup \{0\}$,
it's also possible to define primes of $\Z$.
A prime of $\Z$ is an integer not $-1, 0, 1$
such that if $d \mid p$, then $d = \pm 1$ or $d = \pm p$.
In that case primes of $\Z$ are
$\pm 2, \pm 3, \pm 5, \pm 7, \pm 11, ...$.)


\begin{prop} \textnormal{(Euclid)}
  There are infinitely many primes.
\end{prop}

\proof
TODO


\begin{prop} \textnormal{(Euclid)}
  There are arbitrarily long consecutives integers of composites.
\end{prop}

\proof
$n!+2$,
$n!+3$,
$n!+4$,
$\ldots$,
$n!+n$
are all composities for $n \geq 2$.
This list is therefore a list of consecutive composites of length
$n - 1$.
\qed


The follow lemma is extremely important and is used
for instance in the fundamental theorem of arithmetic to prove
the uniqueness of prime factorization in $\N$ (or $\Z$).
To break tradition, we will call this a theorem (instead of a lemma):

\begin{thm}
  \textnormal{(\defone{Euclid's lemma})}
  If $p$ is a prime and $p \mid ab$, then either $p \mid a$ or $p \mid b$.
\end{thm}

\proof
TODO


The above generalizes easily (by induction) to the following:


\begin{cor}
If $p$ is a prime and $p \mid a_1 a_2 \cdots a_n$, then 
$p$ divides at least one of the $a_1, \ldots, a_n$.
\end{cor}
\proof
We will prove this by strong induction on the number of terms
in $a_1, ..., a_n$.
Hence let $P(n)$ be the above statement, i.e., $P(n)$ is the
statement that if a prime divides the product of $n$ terms,
then $p$ divides at least one of the terms.

The case of $n = 2$ is Euclid's lemma.
Hence $P(2)$ holds.
This is the base case of our induction.

Now assume $P(k)$  is true for $2 \leq k \leq n$.
Let $p$ be a prime divide $a_1 a_2 \cdots a_n a_{n + 1}$.
Therefore $p$ divides $a_1 a_2 \cdots a_{n-1} b$ (there are $n - 1$ terms)
where $b = a_n a_{n + 1}$.
Since $P(n)$ is true, $p$ divides at least one of $a_1, ..., a_{n - 1}, b$.
Since $P(2)$ is true,
$p$ divides $b = a_n a_{n + 1}$ if $p$ divides $a_n$ or $a_{n + 1}$.
Hence $p$ divides at least one of $a_1, ..., a_{n + 1}$.
Therefore $P(n + 1)$ holds.
\qed

\begin{thm} \textnormal{(Euclid's \defone{Fundamental Theorem of Arithmetic})}
Every positive integer $> 1$ can be written as a unique product of primes
up to permutation of the prime factors.
This means
\begin{enumerate}[nosep,label=\textnormal{(\alph*)}]
\item[(a)] If $n > 1$ is an integer, then $n$ can be written as a product
of primes.
\item[(b)] If $n$ is written as two products of primes:
\[
n = p_1 p_2 \cdots p_k = q_1 q_2 \cdots q_\ell
\]
where $p_i$ and $q_j$ are primes arranged in ascending order, i.e.,
\begin{align*}
p_1 \leq p_2 \leq \cdots \leq p_k \\
q_1 \leq q_2 \leq \cdots \leq q_\ell
\end{align*}
then $k = \ell$ and 
\[
p_1 = q_1, \,\,\,\,\,
p_2 = q_2, \,\,\,\,\,
\cdots, \,\,\,\,\,
p_k = q_k, \,\,\,\,\,
\]
\end{enumerate}
\end{thm}

\proof
TODO

The statement of the Fundamental Theorem of Arithmetic can include
the case of $n = 1$ if we accept that the product of an empty
set of integers is $1$:
\[
\prod_{p \in \{\}} p = 1
\]

\begin{prop} Let $a = \prod_{p \in P} p^{a_p}$, $b = \prod_{p
    \in P} p^{b_p}$ and $c = \prod_{p \in P} p^{c_p}$
  where $P$ is a finite set of primes. Then
\begin{enumerate}[nosep,label=\textnormal{(\alph*)}]
 \item $c = ab$ $\implies$ $c_p = a_p + b_p$.
 \item $a \mid b$ $\implies$ $a_p \leq b_p$ for all $p \in P$.
 \item $c = \gcd(a,b)$ $\implies$ $c_p = \min(a_p, b_p)$.
\end{enumerate}
\end{prop}

The above assumes the easily proven facts that
\[
\prod_{p \in P} p^{a_p}
\prod_{p \in P} p^{b_p}
=
\prod_{p \in P} p^{a_p + b_p}
\]
(by commutativity of $\cdot$) and
\[
\prod_{p \in P} p^{a_p}
=
\prod_{p \in P} p^{b_p}
\implies
a_p = b_p \text{ for all $p \in P$}
\]
by uniqueness of prime factorization from the Fundamental Theorem of Arithmetic.
Here, $P$ is a set of distinct primes.


\begin{prop}
  If $n > 1$ is not a prime, then there is a prime factor $p$ such that
  $p \leq \sqrt{n}$.
\end{prop}
\proof
TODO
\qed

Therefore a very simple primality test algorithm for $n$ is the following:
  
\begin{Verbatim}[fontsize=\footnotesize,frame=single]
ALGORITHM: BRUTE-FORCE-PRIMALITY-TEST
INPUT: n
OUTPUT: true if n is prime. If n < 2, false is returned.

if n < 2: return false

d = 2
while d <= sqrt(n):
    if n % d == 0:
        return false
    d = d + 1
return true
\end{Verbatim}

In terms of $n$, the runtime is $O(\sqrt{n})$.
However in terms of the bits of $n$, by Proposition 1.5.2,
the number of $n$ is
\[
b = \floor{\log_2 ( n + 1 )} = \log_2(n + 1) - \alpha
\]
where $0 \leq \alpha < 1$.
Hence
\[
n = 2^b2^\alpha - 1
\]
Hence in terms of the number of bits of $n$,
the runtime is
\[
O(\sqrt{n}) = O(\sqrt{2^b2^\alpha - 1}) = O(2^{b/2})
\]
It is common to denote the number of bits of the input by $n$.
Hence the runtime in number of bits $n$ is
$O(2^{n/2})$, i.e., it has \defone{exponential runtime with linear exponent}.

$O(\sqrt{n})$ is said to be the \defone{pseudo-polynomial runtime} of the
algorithm
to indicate that the $n$ is the numeric input and not a
correct measure of the complexity of the input,
which should be in number of bits of the input.

\begin{ex}
  Let $P = \{p_1, ..., p_n\}$ be a set of distint primes.
  Consider the expression $N(P) = \prod_{p \in P} p + 1$ in Euclid's proof of infinitude of primes. 
  This is sometimes called Euclid's construction.
  How often is this a prime?
\end{ex}

\begin{ex}
  Let $P = \{p_1, ..., p_n\}$ be a set of distint primes.
  Carry out the Euclid's construction on all possible subsets of $P$.
  If a Euclid construction is a prime, put that prime into $P$.
  If it does not, put the smallest prime factor into $P$.
  Repeat.
  For instance if you start with $P = \{\}$,
  you'll get $2$ and the new $P$ is $\{2\}$.
  Next you'll get $P = \{2, 3\}$.
  The Euclid constructions you get from $P$ are $2, 3, 4, 7$.
  So the next $P$ is $\{2, 3, 7\}$.
  At the next stage you get $2, 3, 4, 8, 7, 15, 22, 43$.
  This means that the next $P$ is $\{2, 3, 7, 43\}$.
  Etc.
  Does your $P$ always grow?
  Notice that your $P$'s so far does not capture $5$.
  When, if at all, will $5$ appear?
\end{ex}

\begin{ex}
  In the above, if an Euclid construction does not give you a prime,
  you take the smallest prime factor.
  What if you pick the largest prime factor?
\end{ex}

The following are some DIY exercises for self-study on famous unsolved problems in
number theory.
You might want to write programs to check on the conjectures.

\begin{ex}
  Are there infinitely many primes $p$ such that $p + 2$ is also a prime?
  If $p$ and $p + 2$ are both primes, then they are called \defone{twin primes}.
  The \defone{twin prime conjecture} states that there are infinitely many
  twin primes.
  (See \url{https://en.wikipedia.org/wiki/Twin_prime}.)
  Write a program that prints $p,p+2$ if both are primes.
  Print the time elapsed between the discovery of pairs of twin primes. 
\end{ex}

\begin{ex}
  Can even positive integer be written as the sum of two primes?
  When the two primes are $> 2$, then the sum of these two primes
  is even.
  The \defone{Goldbach conjecture} states that
  every positive integer can be written as the sum of two primes.
  (See \url{https://en.wikipedia.org/wiki/Goldbach%27s_conjecture}.)
  Write a program that prints $n \geq 2$ as a sum of two primes as $n$
  iterates from 2 to a huge positive integer $N$ entered by the user.  
\end{ex}

\begin{ex}
  A positive integer is a \defterm{perfect} number if it is the
  sum of its positive divisors strictly less than itself.
  For instance $6$ is perfect since $6 = 1 + 2 + 3$.
  $28$ is also perfect since $28 = 1 + 2 + 4 + 7 + 14$.
  Euclid knew that if $p$ is prime and $2^p - 1$ is a prime,
  then $2^{p-1}(2^p - 1)$ is an even perfect number.
  If $p$ is a prime and $2^p - 1$ is a also a prime,
  then $2^p - 1$ is called a \defterm{Mersenne prime}.
  Almost 2000 years after Euclid, Euler proved the converse, that
  every even perfect number must be of the form
  $2^{p-1}(2^p - 1)$ where $2^p - 1$ is a Mersenne prime.
  There are two famous unsolved problems in number theory on perfect
  numbers:
  \begin{enumerate}[nosep]
  \item Are there odd perfect numbers?
  \item Are there infinitely many perfect numbers?
  \end{enumerate}
  (See \url{https://en.wikipedia.org/wiki/Perfect_number}.)
  There is an ongoing search for primes and Mersenne primes using
  computers.
  At this point (2023), the top 8 largest known primes are all
  Mersenne primes.
  (See \url{https://en.wikipedia.org/wiki/Great_Internet_Mersenne_Prime_Search}.)
  As of Feb 2023, the largest known prime is $2^{82,589,933} - 1$.
  (See \url{https://en.wikipedia.org/wiki/Largest_known_prime_number}.)
  Write a program that prints perfect numbers, printing the time between pairs of
  perfect numbers discovered.
\end{ex}


\begin{ex}
  Leetcode 204 \\
  \url{https://leetcode.com/problems/count-primes/} \\
  Given an integer $n$, return the number of prime numbers that are strictly
  less than $n$.
\end{ex}

\begin{ex}
  Leetcode 507 \\
  \url{https://leetcode.com/problems/perfect-number/} \\
  A perfect number is a positive integer that is equal to the sum of its
  positive divisors, excluding the number itself.
  A divisor of an integer \verb!x! is an integer that can divide \verb!x!
  evenly.
  Given an integer \verb!n!, return \verb!true! if \verb!n! is a perfect number,
  otherwise return \verb!false!.
\end{ex}

\begin{ex}
  Leetcode 866 \\
  \url{https://leetcode.com/problems/prime-palindrome/}\\
  Given an integer $n$, return the smallest prime palindrome greater than or
  equal to $n$.
  An integer is prime if it has exactly two divisors: 1 and itself.
  Note that 1 is not a prime number.
  For example, 2, 3, 5, 7, 11, and 13 are all primes.
  An integer is a palindrome if it reads the same from left to right
  as it does from right to left.
  For example, 101 and 12321 are palindromes.
  The test cases are generated so that the answer always exists and is
  in the range $[2, 2 * 108]$.
\end{ex}

\begin{ex}
  Leetcode 1175 \\
  \url{https://leetcode.com/problems/prime-arrangements/}
  Return the number of permutations of 1 to $n$ so that prime numbers are at
  prime indices (1-indexed.)
  (Recall that an integer is prime if and only if it is greater than 1,
  and cannot be written as a product of two positive integers both smaller
  than it.)
  Since the answer may be large, return the answer modulo $10^9 + 7$.
\end{ex}

\begin{ex}
  Leetcode 1362 \\
  \url{https://leetcode.com/problems/closest-divisors/} \\
  Given an integer \verb!num!,
  find the closest two integers in absolute difference whose product equals
  \verb!num + 1! or \verb!num + 2!.
  Return the two integers in any order.
\end{ex}

\begin{ex}
  Leetcode 1390 \\
  \url{https://leetcode.com/problems/four-divisors/} \\
  Given an integer array \verb!nums!,
  return the sum of divisors of the integers in that array that have exactly
  four divisors.
  If there is no such integer in the array, return \verb!0!.
\end{ex}

\begin{ex}
  Leetcode 2523 \\
  \url{https://leetcode.com/problems/closest-prime-numbers-in-range/} \\
  Given two positive integers \verb!left! and \verb!right!, find the two integers \verb!num1! and \verb!num2! such that:
  \begin{enumerate}
  \li \verb!left <= nums1 < nums2 <= right!
  \li \verb!nums1! and \verb!nums2! are both prime numbers.
  \li \verb!nums2 - nums1! is the minimum amongst all other pairs satisfying the above conditions.
  \end{enumerate}
  Return the positive integer array \verb!ans = [nums1, nums2]!.
  If there are multiple pairs satisfying these conditions,
  return the one with the minimum nums1 value or \verb![-1, -1]! if such numbers do not exist.
  A number greater than 1 is called prime if it is only divisible by 1 and itself.
\end{ex}

\begin{ex}
  Leetcode 2521 \\
  \url{https://leetcode.com/problems/distinct-prime-factors-of-product-of-array/} \\
  Given an array of positive integers \verb!nums!,
  return the number of distinct prime factors in the product of the elements of \verb!nums!.
\end{ex}

\begin{ex}
  Leetcode 1071 \\
  \url{https://leetcode.com/problems/greatest-common-divisor-of-strings/} \\
  For two strings $s$ and $t$, we say
  \lq\lq $t$ divides $s$" if and only if $s = t + \cdots + t$
  (i.e., $t$ is concatenated with itself one or more times).
  Given two strings \verb!str1! and \verb!str2!, return the largest string
  \verb!x! such that \verb!x! divides both \verb!str1! and \verb!str2!.
\end{ex}
