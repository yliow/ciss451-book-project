\section{The Prime Number Theorem and finding primes}

We'll need a way to create huge primes for RSA.

Of course you know Euclid's theorem that says
there are infinitely many primes.
So we don't have to worry about not finding them.
But just because there are infinitely many primes, it does not mean
they are everywhere!

Generally, we want to specify how large our primes should be.
This is specified by the bit length of the primes.
Given the length, one can generate a sequence of bit of that length.
Of course that number should be odd.
So the least significant bit is set to $1$.
Call this $n$.
One can than check if $n$ is a prime.
If $n$ is not prime, we try $n + 2$, etc.

Does this process take a long time?

Gauss was the first to realize that even though primes seem to appear
randomly on the real line,
\input{primes.tex}
the density of primes seems to follow some
law.
The density of primes can be defined this way.
Let $\pi(x)$ be the number of primes $\leq x$.
Then the density of primes up to $x$ is
\[
\frac{\pi(x)}{x}
\]
Here is the plot of $\pi(x)$ up to $x = 20$:
\input{pi_x_20.tex}
When the plot is up to $x = 100$ one begin to see that the
rough and jagged graph begin to smooth out:
\input{pi_x_100.tex}

Through analyzing tables of primes, Gauss discovered that
$\pi(x)/x$ is \defone{asymptotically equivalent} to $1/\ln x$ ($\ln = \log_e$)
\[
\frac{\pi(x)}{x} \sim \frac{1}{\ln x}
\]
i.e.,
\[
\lim_{x \rightarrow \infty} \frac{\pi(x)/x}{1/\ln x} = 1
\]
Equivalently
\[
\pi(x) \sim \frac{x}{\ln x}
\]
Here a plot of $\pi(x)$ and $x/\ln x$ up to $x = 10^5$:
\input{pi_x_100000.tex}

The above was first conjectured by Gauss in 1782/3
and finally proven in 1896 by
\href{https://en.wikipedia.org/wiki/Jacques_Hadamard}{Hadamard}
and
\href{https://en.wikipedia.org/wiki/Charles_Jean_de_la_Vall%C3%A9e_Poussin}{de la Vallée Poussin}:

\begin{thm} \textnormal{(\defone{Prime Number Theorem})}
\[
\pi(x) \sim \frac{x}{\ln x}
\]  
\end{thm}

Among number theorists and researchers in cryptography,
the above deep result is known as \defone{PNT}.

By PNT, when $x$ is large, the density of primes up to $x$ is approximately
$1/\ln x$.
Using the sieve of Eratosthenes, the number of primes up to $x = 10^5$ is
$9592$, i.e., $\pi(x)/x = 9592/100000 = 0.09592 = 9.592\%$
which is very close to $1/\ln x = 1/\ln 10^5 = 0.086858... = 8.6858...\%$.
If we search for a prime only among \textit{odd} integers, the
chance of finding a prime is $2/\ln x = 0.173717... = 17.3717...\%$.

For modern-day RSA, primes used have approximately 1024-2048 bits.
If we choose a bit length of 1024, then
$2/\ln 2^{1024} = 0.1408...\%$.
Therefore one might find a prime after $< 1000$ tries among odd integers.
Usually one would begin with an integer $n$
with a random sequence of 1024 bits, with least
significant bit being 1 (so that $n$ is odd).
Then a primality test is used to check if $n$ is prime.
We'll see that a probabilistic primality test is used.
If $n$ is not a prime, one would then try $n + 2$. Etc.

Next we will look at two very important primality tests: Fermat primality test
and Miller--Rabin primality test.
Miller--Rabin primality test is the one that is used in the real world.
However the main idea in Miller--Rabin primality test is
actually Fermat primality test.

\begin{ex}
  Write a function \verb!rand_odd_int! that accept $L$ and return an odd
  positive integer with $L$ random bits.
  Hint: For python, try this:
  \begin{Verbatim}[frame=single,fontsize=\footnotesize]
n = int("0b111", 2) # the "0b" is optional
print(n)
n = int("0b100", 2)    
print(n)
  \end{Verbatim}
  Try a few more examples to understand what is happening.
\end{ex}

\begin{ex}
  Write a function \verb!eratosthenes! that accepts an integer $n$
  and returns a bool array \verb!isprime! of size $n$ such that
  \verb!isprime[i]! is True iff \verb!i! is prime.
\end{ex}

\begin{ex}
  Write a function \verb!primes! that accepts $x$
  and returns an array of primes from $2$ up to $x$ (inclusive) in
  ascending order.
  For instance \verb!primes(10)! return \verb![2, 3, 5, 7]!.
\end{ex}

\begin{ex}
  Write a function \verb!write_primes! that accepts \verb!x! and a path \verb!p!
  and store primes up to \verb!x! at path \verb!p! in comma-separated format.
  For instance \verb!write_primes(10, 'primes-10.txt')!
  writes \verb!"2,3,5,7"! to the file \verb!primes-10.txt!.
  Write another function \verb!read_primes! that accepts a path \verb!p!
  and returns a list of primes stored at path \verb!p!.
  Create a file of primes up to 10,000,000.
  After you are done with the above make a slight optimization by storing
  integer in hex.
  While a decimal (base-10 digit) can store 10 patterns,
  a hexadecimal can store 16.
  Try this:
  \begin{Verbatim}[frame=single,fontsize=\footnotesize]
i = int("0x1a", 16) # the "0x" is optional
print(i)
s = hex(i)
print(s)
  \end{Verbatim}
  (It's even better to store the integer directly in binary format, but
  that makes the file non-human readable.)
\end{ex}

\begin{ex}
  Write a function \verb!pi! that accept \verb!x! and returns
  the number of primes up to \verb!x! (inclusive).
\end{ex}


