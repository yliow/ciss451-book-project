\section{Definitions}

The most basic mathematical object is $\Z$.
$\Z$ has two operations: addition and multiplication.
We first abstract the study of $\Z$ by focusing on just one operation, the $+$.

\begin{defn}
  $(G, *, e)$ is a
  \defone{group}
  if $G$ is a set and $*$ satisfies
  \begin{enumerate}[nosep]
  \item[(C)] If $x, y \in G$, then $x * y \in G$. In other words $*: G \times G \rightarrow G$ is a binary operator.
  \item[(A)] If $x, y, z \in G$, then $(x * y) * z = x * (y * z)$.
  \item[(I)] If $x\in G$, then there is some $y \in G$ such that $x * y = e = y * x$.
    $y$ is called an \defone{inverse} of $x$.
    Later we will see that the inverse of $x$ is uniquely determined by $x$.
  \item[(N)] If $x \in G$, then $x * e = x = e * x$.
  \end{enumerate}
\end{defn}

\begin{defn}
  $(G, *, e)$ is an \defone{abelian group} if $(G, *, e)$ is a group such that
  if $x,y \in G$, then $x * y = y * x$.
  In other words, $(G, *, e)$ is an abelian group if
  $(G, *, e)$ is group and $*$ is a commutative operator.
\end{defn}

The reason for now including the commutativity condition in the definition for groups is because
there are many important groups which are not abelian.
