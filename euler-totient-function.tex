\section{Euler Totient Function}

\begin{defn}
Let $N$ be a positive integer. $\phi(N)$ is the number of positive
integers from $0$ to $N-1$ which are coprime to $N$, i.e.
\[
\phi(N) = | \{ a \,|\, 0 \leq a \leq N - 1, \,\,\,\gcd(a,N) = 1\} |
\]
\end{defn}

Note that you can also view $\phi$ in this way:
\begin{align*}
\phi(N) 
&= | \{ a \mid 0 \leq a \leq N - 1, \,\,\,\gcd(a,N) = 1\} | \\
&= | \{ a \mid 0 \leq a \leq N - 1, \,\,\, a \text{ is invertible mod } N\}|\\
&= | (\Z/N^\times) |
\end{align*}
Recall that
$\{ a \,|\, 0 \leq a \leq N - 1, \,\,\,\gcd(a,N) = 1\}$
is the set of units of $\Z/N$ which is denoted by $(\Z/N)^\times$ or $U(\Z/N)$.
So the above definition is the same as
\[
\phi(N) = | U(\Z/N) |
\]

Note that by definition $\phi(1) = 1$.
Here are some important properties of $\phi$.

\begin{prop} Let $n > 0$ be a positive integer.
  \begin{enumerate}[topsep=0in, nosep]
  \item[\textnormal{(a)}]
    Then
    \[
    \phi(n) = n \prod_{p \mid N} \left( 1 -  \frac{1}{p} \right)
    \]
    where \lq\lq $\prod_{p \mid n}$"
    is
    \lq\lq product over all primes $p$ dividing $n$".
  \item[\textnormal{(b)}] If $m,n$ are coprime, i.e. $\gcd(m,n)=1$, then
    $\phi(mn) = \phi(m)\phi(n)$.
  \item[\textnormal{(c)}]
    If $p$ is a prime and $k > 0$, then
    $\phi(p^k) = p^{k-1}(p-1) = p^k - p^{k-1}$.
  \end{enumerate}
\end{prop}
\proof
TODO
\qed

Let's compute $\phi(10)$.
Note that $10 = 2 \cdot 5$ and $\gcd(2,5) = 1$.
Therefore using (b) of the above theorem we get
\[
\phi(10) = \phi(2^1 \cdot 5^1) = \phi(2^1) \cdot \phi(5^1)
\]
since $\gcd(2^1 \cdot 5^1) = 1$.
Using (c) of the above theorem I get
\[
\phi(10) = \phi(2^1) \cdot \phi(5^1)
= (2^1 - 2^{1-1}) \cdot (5^1 - 5^{1-1}) = 1 \cdot 4 = 4
\]
Of course you can also use (a) above to get
\[
\phi(10) = 10 \cdot (1 - 1/2) \cdot (1 - 1/5)
= 10 \cdot \frac{1}{2} \cdot \frac{4}{5} = 4
\]
In both cases, we get 4.
This means in $\Z/10$, there are 4 elements which are invertible.
Running $a$ through $\{0, 1, 2, ..., 9\}$,
you'll see that invertible $a$'s are $1, 3, 7, 9$
with inverses $1, 7, 3, 9$ respectively.

\begin{ex}
  \mbox{}
  \begin{enumerate}[nosep,label=\textnormal{(\alph*)}]
  \item  Compute $\phi(735)$.
  \item  Compute $\phi(900)$.
  \item  Compute $\phi(263891)$.
  \end{enumerate}
\end{ex}

\begin{ex}
  \begin{enumerate}[nosep,label=\textnormal{(\alph*)}]
  \item  Let $p$ be a prime. What is $\phi(2p)$ in terms of $p$?
  \item  How many solutions are there to $\phi(n) = 2n$?
  \item  How many solutions are there to $\phi(n) = n/2$?    
  \end{enumerate}
\end{ex}


\begin{ex}
  Easy: What is $\phi(pq)$ as an integer expression
  involving $p$ and $q$? Can you write it as an expression
  involving the sum and product of $p$ and $q$? (i.e., besides
  constants and operators, your 
  expression contains only $p+q$ and $pq$).
\end{ex}


\begin{ex}
  \mbox{}
  \begin{enumerate}[nosep]
    \item[(a)]
      Solve $\phi(n) = 2$, i.e., find all positive integers $n$ such that
      $\phi(n) = 2$.
      (Hint: Write down the prime factorization of
      $n = p_1^{e_1}\cdots p_g^{e_g}$ and use the
      equation $\phi(n) = 2$.)
    \item[(b)] Solve $\phi(n) = 3$.
    \item[(b)] Solve $\phi(n) = 6$.
  \end{enumerate}
\end{ex}


\begin{ex}
  Prove that
  \[
  \phi(mn) = \phi(m)\phi(n) \cdot \frac{g}{\phi(g)}
  \]
  where $g = \gcd(m,n)$.
  (Note that the above does \textit{not} assume $m,n$ are coprime.
  What is the above if $m,n$ are coprime?)
\end{ex}


\begin{ex} $^*$
Plot a function of the graph $y = \phi(x)$ for integer values of $x$
running through $1$ to $10000$.
See any pattern? Note that if you want an approximation (for instance
in the asymptotics) that's not too difficult.
Note the following:
There are two ways to compute $\phi(n)$:
\begin{enumerate}[nosep,label=\textnormal{(\alph*)}]
\item $\phi(n) = |\{x \mid 0 \leq x < n - 1, \,\,\, \gcd(x, n) = 1\}|$
  which required Euclidean algorithm in a loop.
\item
  $\phi(n)$
  $= \phi(p_1^{\alpha_1} \cdots p_g^{\alpha_g})$
  $= (p_1^{\alpha_1} - p_1^{\alpha_1 - 1}) \cdots (p_1^{\alpha_g} - p_1^{\alpha_g - 1})$
  which requires prime factorization.
\end{enumerate}
The first method is slow: you need to loop your $x$ and for each $x$ you need to
execute the EEA which has a loop.
The second method is fast only if you can find the prime
factorization of $n$.
Is it possible to find $\phi(n)$ as a formula in $n$ without finding the
prime factorization of $n$?
For instance the factorial function $n!$ has this interpolation:
If
\[
\Gamma(x) = \int_0^\infty t^{x-1} e^{-t} \ dt
\]
then $\Gamma(n) = n!$ for positive integers $n$.
Is there a fast interpolation of $\phi(n)$? 
\end{ex}


\begin{ex}$^*$
  Can you find an $n$ such that $\phi(n)$ divides $n + 1$?
\end{ex}


\begin{ex}$^*$
  \begin{enumerate}[nosep,label=\textnormal{(\alph*)}]
  \item If $p$ is a prime, then $\phi(p) = p - 1$.
    Prove that if $\phi(n) = n - 1$, then $n$ is a prime.
  \item
    Can you find an $n > 1$ which is not a prime (i.e. composite)
    and such that
    $\phi(n)$ divides $n - 1$?
    If you can find one, let me know ASAP.
    Or if you can prove that such as $n$ does not exist, let me know ASAP.
  \end{enumerate}
\end{ex}

\begin{ex}$^*$
  \mbox{}
  \begin{enumerate}[nosep,label=\textnormal{(\alph*)}]
    \item
      Can you find some $n$ such that
      \[
      \phi(\phi(n)) = 1
      \]
    \item
      Write $\phi^2(n) = \phi(\phi(n))$.
      Can you find \textit{all} $n$ such that $\phi^2(n) = 1$.
    \item
      Write $\phi^k$ to the composition of $k$ Euler $\phi$.
      What about $\phi^3(n) = 1$?
      Can you find some $n$ satisfying the above equation?
    \item
      What about $\phi^k(2^n) = 1$? What is the smallest $k$
      for $\phi^k(2^n) = 1$?
    \item
      What about $\phi^k(2^m \cdot 3^n) = 1$? What is the smallest $k$
      such that $\phi^k(2^m \cdot 3^n) = 1$?
  \end{enumerate}
\end{ex}


As an aside, note that $\phi$ as a function has domain of $\N$.
In this case, we say that $\phi$ is an
\defterm{arithmetic function}\tinysidebar{arithmetic function}\index{arithmetic function}.
Furthermore, $\phi$ satisfies the property that
if $\gcd(m,n) = 1$, then $\phi(mn) = \phi(m)\phi(n)$.
A function $\N \rightarrow \C$ satisfying this property
is said to be
\defterm{multiplicative}
\index{multiplicative}
\tinysidebar{multiplicative}.
The Euler $\phi$ function is one of many multiplicative arithmetic
functions.
Multiplicative functions are extremely important in number theory.

The following theorem allows you to compute powers in mod $p$ extremely
fast:

\begin{thm} \textnormal{(\defone{Fermat's Little Theorem})}.
  Let $p$ is a prime number
  and $a$ be a positive integer not divisible by $p$. Then
  \[
  a^{p-1} \equiv 1 \,\,\,(\operatorname{mod} p)
  \]
\end{thm}


\proof
TODO
\qed


\begin{ex}
Compute $r$ where $r$ is the smallest positive integer satisfying
\[
 5^{642} \equiv r \,\,\,(\operatorname{mod} 641)
\]
\end{ex}

\begin{cor}
Let $p$ be a prime. Then $a^p \equiv a
\,\,\,(\operatorname{mod} p)$.
\end{cor}

Note that the corollary does not require $p \nmid a$.
The proof of the corollary is easy.
If $p \mid a$, then both sides of the equation is $0$ mod $p$,
so the congruence is true.
If $p \nmid a$, then Fermat's Little Theorem gives us
\[
a^{p-1} \equiv a \pmod{p}
\]
on multiplying both sides by $a$, we get
\[
a^p \equiv a \pmod{p}
\]

\begin{ex}
What is the remainder of $3^{122436481}$ mod 13?
\end{ex}


Note that Fermat's Little Theorem can be used to compute powers
very rapidly if you work in mod $p$ where $p$ is a prime. What if
you need to work in mod $N$ where $N$ is not a prime? There is a
generalization of Fermat's Little Theorem due to Euler. Note that
since $p-1 = \phi(p)$, Fermat's Little Theorem
\[
 a^{p-1} \equiv 1 \,\,\,(\operatorname{mod} p)
\]
can be stated as
\[
 a^{\phi(p)} \equiv 1 \,\,\,(\operatorname{mod} p)
\]
This statement actually holds if $p$ is replaced by any positive
integer.

\begin{thm} \textnormal{(\defone{Euler's Theorem})}.
  Let $a$ and $n$ be positive integers such that $\gcd(a, n) = 1$.
  Then
  \[
  a^{\phi(n)} \equiv 1 \,\,\,(\operatorname{mod} n)
  \]
\end{thm}
\proof
TODO
\qed

Note that for $\Z/N$, a computation of
\[
a^k \pmod{N}
\]
Euler's Theorem can lower the $k$ if $\gcd(a, N)$.
But after you have lowered the $k$, say to $\ell$,
you still need to compute $a^\ell \pmod{N}$.
See the squaring algorithm in the RSA chapter.



\begin{ex}
Compute $r$ where $r$ is the smallest positive integer satisfying
\[
 5^{642} \equiv r \,\,\,(\operatorname{mod} 640)
\]
\end{ex}


\begin{ex}
What is the remainder of $3^{123456789}$ mod 100?
\end{ex}

\begin{ex}
What is the hundreds digit of $3^{123456789}$?
\end{ex}

\begin{ex}
  In your \verb!Zmod.py! complete the following:
  \begin{enumerate}[nosep]
    \li Exponentiation (i.e., \verb!__pow__!)
  \end{enumerate}
  Note that $x^{-1000} \pmod{N}$ is $(x^{-1})^{1000} \pmod{N}$.
  You want to first check if you can use Euler's Theorem.
  If it is, use the theorem to lower the exponent to say $\ell$.
  Then use the obvious loop to compute $x^\ell$ and apply $\pmod{N}$
  as frequently as possible.
  After you are done with the above,
  you might want to improve your \texttt{\_\_pow\_\_} by
  \textit{not} use Euler's theorem
  if the exponent is \lq\lq small".
  You can determine for yourself what if \lq\lq small".
  For instance you can choose to use Euler's Theorem
  only when the exponent is greater than $10$.
  (Later in the RSA chapter, we will talk about the squaring method.)
\end{ex}

\begin{ex}
  Leetcode 372.\\
  \url{https://leetcode.com/problems/super-pow/} \\
  Your task is to calculate $a^b \pmod{1337}$ where $a$ is a positive integer
  and $b$ is an extremely large positive integer given in the form of an array.
  For instance
  for $a = 2, b = [1,0]$,
  the output is $1024$.
\end{ex}

\begin{ex}
  Leetcode 1015.\\
  \url{https://leetcode.com/problems/smallest-integer-divisible-by-k/}\\
  Given a positive integer $k$, you need to find the length of the smallest positive integer
  $n$ such that $n$ is divisible by $k$, and $n$ only contains the digit $1$.
  Return the length of $n$. If there is no such $n$, return $-1$.
  Note: $n$ may not fit in a 64-bit signed integer.
\end{ex}

\begin{ex}
  Leetcode 1622.\\
  \url{https://leetcode.com/problems/fancy-sequence/}\\
  Write an API that generates fancy sequences using the append, addAll, and multAll operations.
  Implement the Fancy class:
  \begin{enumerate}[nosep]
    \li \verb!Fancy()! Initializes the object with an empty sequence.
    \li \verb!void append(val)! Appends an integer val to the end of the sequence.
    \li \verb!void addAll(inc)! Increments all existing values in the sequence by an integer inc.
    \li \verb!void multAll(m)! Multiplies all existing values in the sequence by an integer \verb!m!.
    \li \verb!int getIndex(idx)! Gets the current value at index \verb!idx! (0-indexed)
    of the sequence modulo 109 + 7.
    If the index is greater or equal than the length of the sequence, return \verb!-1!.
  \end{enumerate}
\end{ex}

\begin{ex}
  Leetcode 1952\\
  \url{https://leetcode.com/problems/three-divisors/}\\
  Given an integer \verb!n!, return \verb!true! if \verb!n! has exactly three
  positive divisors. Otherwise, return \verb!false!.
  An integer \verb!m! is a divisor of \verb!n!
  if there exists an integer \verb!k! such that \verb!n = k * m!.
\end{ex}


\begin{ex}
  \url{https://acm.timus.ru/problem.aspx?space=1&num=1673}\\
  At the end of the previous semester the students of the
  Department of Mathematics and Mechanics of the Yekaterinozavodsk State
  University had to take an exam in network technologies.
  $N$ professors discussed the curriculum and decided that there would be
  exactly $N^2$ labs, the first professor would hold labs with numbers
  $1, N + 1, 2N + 1, ..., N^2 - N + 1$, the second one — labs with numbers
  $2, N + 2, 2N + 2, ..., N^2 - N + 2$, etc.
  $N$-th professor would hold labs with numbers
  $N, 2N, 3N, ..., N^2$.
  The professors remembered that during the last years lazy students didn't
  attend labs and as a result got bad marks at the exam.
  So they decided that a student would be admitted to the exam only if
  he would attend at least one lab of each professor.
  $N$ roommates didn't know the number of labs and professors in this semester.
  These students had different diligence: the first student attended all labs,
  the second one — only labs which numbers were a multiple of two,
  the third one — only labs which numbers were a multiple of three, etc…
  At the end of the semester it turned out that only $K$ of these students
  were admitted to the exam. Find the minimal $N$ which makes that possible.

  Input:
  An integer $K$ ($1 \leq K \leq 2 \cdot 10^9$).
  
  Output:
  Output the minimal possible N which satisfies the problem statement.
  If there is no $N$ for which exactly $K$ students would be admitted to the
  exam, output 0.

  Example:
  Input:8, output:15.
  Input:3, output:0.

\end{ex}
