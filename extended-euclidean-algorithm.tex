\section{Extended Euclidean algorithm: GCD as linear combination}

Here's another super important fact:

\begin{thm}
  \textnormal{(\defterm{Extended Euclidean Algorithm}\index{Extended Euclidean Algorithm}\tinysidebar{Extended Euclidean Algorithm})}
If $a$ and $b$ be integers which are not both zero,
then there are integers $x,y$ such that
\[
\gcd(a,b) = ax + by
\]
\end{thm}

The $x,y$ in the above theorem are called
\defterm{B\'ezout's coefficients}\index{B\'ezout's coefficients}\tinysidebar{B\'ezout's coefficients}
of $a,b$.
They are not unique.

\begin{ex}
  Prove that $a \neq 0$, then
  there are many possible $x, y$ such that
  $ax + by = \gcd(a,b)$.
  \qed
\end{ex}
% ax + by = ax + by + ab - ab = a(x + b) + b(y - a).
% ax + by = ax + by + nab - nab = a(x + nb) + b(y - na).


First let me prove that there are $x,y$ such that
\[
ax + by = \gcd(a, b)
\]
The theorem does not give you the algorithm.
Then I'll do a computational example
that compute the gcd of $a,b$ as a linear combination of $a$ and $b$.
The example actually contains the idea behind the algorithm
to compute the B\'ezout's coefficients.
The algorithm is called
the Extended Euclidean Algorithm.

\proof
For convenience, let
me write $(a, b)$ as the set of linear combinations of $a$ and $b$, i.e.,
\[
(a, b) = \{ax + by \mid x,y \in \Z \}
\]
I will also write $(g)$ for the linear combination of $g$, i.e.,
\[
(g) = \{gx \mid x \in \Z \}
\]
(Such linear combinations are called ideals. They are extremely
important in of themselves. Historically, they were created
to study Fermat's last theorem. Since then they are crucial in the
the study of ring theory.)

The proof proceeds in two steps:
\begin{enumerate}[nosep]
\item Given $a,b$ not both zero, there is some $g$ such that
  $(a,b) = (g)$. If $a,b$ are not both zero, $g$ can be chosen to be
  $>0$.
\item The $g$ in the above is in fact $\gcd(a,b)$. 
\end{enumerate}

Now let's prove step 1, i.e., given $a,b$, 
there is some $g$ such that
\[
(a, b) = (g)
\]
First of all if $b = 0$, then by definition
\[
(a, 0) = (a)
\]
and we're done.
Next, we now assume $b \neq 0$.
Then $|b| > 0$.
The set
\[
X = \{ax + by \mid x,y \in \Z \text{ and } ax + by > 0 \} \subseteq \N
\]
is nonempty since it contains
$0\cdot a + 1\cdot |b|$.
By the well-ordering principle of $\N$, $X$ has a minimum element, say $g$.
I will now show that $(a, b) = (g)$.

Since $g$ is a minimum element of $X$, $g$ is in $X$.
Therefore $g = ax + by$.
Hence $gz = a(xz) + b(yz) \in (a, b)$ for all $z \in \Z$.
This implies that $(g) \subseteq (a, b)$.

Now I will prove that $(a, b) \subseteq (g)$.
Let $c \in (a, b)$, i.e., $c = ax + by$ for some $x,y \in \Z$.
Therefore by the Euclidean property of $\Z$, there exists $q,r \in \Z$ such that
\[
c = gq + r, \,\,\, 0 \leq r < g
\]
(Look at the definition of $X$ again. $X$ is a subset of $\N$ so that
$g \geq 1$).
If $r \neq 0$, then
\[
r = c - gq
\]
Note that $c = ax + by$ by our assumption.
We have already shown that $(g) \subseteq (a, b)$, i.e., $g = ax' + by'$.
Therefore, altogether we have
\[
r = c - gq = ax + by - (ax' + b'y)q = a(x-x'q) + b(y-y'q)
\]
Hence $r \in X$.
But $0 \leq r < d$ implies that
\[
r = a(x-x'q) + b(y-b'q)
\]
is an element of $X$ which is less than $g$ which contradicts the
minimality of $g$.
Hence $r = 0$ and we have
\[
c = gq + r = gq \in (g)
\]
I have shown that $(a,b) \subseteq (g)$.

Altogether, I've shown $(a,b) = (g)$.
Step 1 is now completed.

For step 2, I 
will show that $g$ is the gcd of $a$ and $b$.
Since $(a,b) = (g)$, we have
\[
a \in (a,b) = (g)
\]
i.e., $a = xg$ which means $g$ divides $a$.
Likewise $g$ divides $b$.
Hence $g$ is a common divisor of $a$ and $b$.
Since $(g) \subseteq (a, b)$,
$g = ax_0 + by_0$.
Suppose $d$ is any divisor of $a$ and $b$.
Then $d \mid ax_0 + by_0$ by the linearity of divisibility.
Hence $d \mid g$.
Therefore $g$ is the largest common divisor of $a$ and $b$,
i.e., $g = \gcd(a,b)$.
\qed

The above does not give you an algorithm to compute the
$x$ and $y$.
First let me do an example to show you that
it's possible to compute $\gcd(a,b)$ as a linear combination of $a$ and $b$.
Then I'll give you the algorithm.

Recall that I have computed $\gcd(514, 24) = 2$.
Extended Euclidean Algorithm says that it's possible to find $x$ and $y$
such that
\[
2 = \gcd(514, 24) = 514x + 24y
\]
How do we compute the $x$ and $y$?
Just like the gcd computation (the Euclidean Algorithm),
the $x,y$ are computed using the Euclidean property.
First we have
\begin{align*}
514 = 21 \cdot 24 + 10
\end{align*}
This implies that 
\[
514 \cdot 1 + 24 \cdot (-21) = 10
\]
Now it would be nice if the pesky $10$ goes away and is replaced by
$2$.
How would we do that?
Well look at $24$ and $10$ now.
We have
\[
24 = 2 \cdot 10 + 4
\]
again by Euclidean algorithm.
Multiplying the equation
\[
514 \cdot 1 + 24 \cdot (-21) = 10
\]
throughout by 2 gives us
\[
514 \cdot 2 + 24 \cdot (-42) = 2 \cdot 10
\]
The previous equation
\[
24 = 2 \cdot 10 + 4
\]
say that $2 \cdot 10$ can be replaced by $24 - 4$.
This means that
\[
514 \cdot 2 + 24 \cdot (-42) = 24 - 4
\]
Hmmm ... this says that we have now
\[
514 \cdot 2 + 24 \cdot (-43) = -4
\]
or
\[
514 \cdot (-2) + 24 \cdot 43 = 4
\]
What about 4?
Well, if we look at $10$ and $4$ just like what we did with $24$ and $10$
we would get
\[
10 = 2 \cdot 4 + 2
\]
and the remainder $2$ gives us the GCD!!!
Rearranging it a bit we have
\[
1 \cdot 10 + (-2) \cdot 4 = 2
\]
i.e. 2 is a linear combination of 10 and 4.
But earlier we say that 4 is a linear combination of 514 and 24 ...
\[
514 \cdot (-2) + 24 \cdot 43 = 4
\]
and even earlier we saw that 10 is also a linear combination of 514 and 24 ...
\[
514 \cdot 1 + 24 \cdot (-21) = 10
\]
Surely if we substitute all these values into the equation
\[
1 \cdot 10 + (-2) \cdot 4 = 2
\]
we would get 2 as a linear combination of 514 and 24.
Let's do it ...
\begin{align*}
2 
&= 1 \cdot 10 + (-2) \cdot 4 \\
&= 1 \cdot (514 \cdot 1 + 24 \cdot (-21)) + (-2) (514 \cdot (-2) + 24 \cdot 43) \\
&= 514 \cdot 1 + 24 \cdot (-21) + 514 \cdot 4 + 24 \cdot (-86) \\
&= 514 \cdot 5 + 24 \cdot (-107)
\end{align*}
V\'oila!

\newpage
\begin{ex}
  Using the above idea,
  compute the $\gcd$ and B\'ezout's coefficients of 210 and 78, i.e.,
  compute $x$ and $y$ such that $210x + 78y = \gcd(210, 78)$.
\end{ex}

\newpage
\begin{ex}
Analyze the above and design an algorithm 
so that when given $a$ and $b$, the algorithm computes
$x$ and $y$ such that $ax + by = \gcd(a,b)$.
\end{ex}

\newpage
To help you analyze the above computation,
let me organize our computations a little.
If we can make the process systematic, then there is hope that we can 
make the idea work for all $a$ and $b$, i.e., then we would have an
algorithm and hence can program it and compute it's runtime performance.

We know for sure that we have to continually use
Euclidean property on pairs of numbers.
So here we go:
\begin{align*}
514 &= 21 \cdot 24 + 10 \\
24  &= 2 \cdot 10 + 4 \\
10  &= 2 \cdot 4 + 2 \\
4   &= 2 \cdot 2 + 0
\end{align*}
Note that this corresponds to the $\gcd$ computation
\begin{align*}
\gcd(514, 24) 
&= \gcd(24, 514 - 21 \cdot 24) = \gcd(24, 10) \\
&= \gcd(10, 24 - 2 \cdot 10) = \gcd(10, 4) \\
&= \gcd(4, 10 - 2 \cdot 4) = \gcd(4, 2)  \\
&= \gcd(2, 4 - 2\cdot 2) = \gcd(2, 0) \\
&= 2
\end{align*}
So in the computation
\begin{align*}
514 &= 21 \cdot 24 + 10 \\
24  &= 2 \cdot 10 + 4 \\
10  &= 2 \cdot 4 + 2 \\
4   &= 2 \cdot 2 + 0
\end{align*}
if the remainder is $0$ (see the last line), 
then the previous line's remainder must be the gcd.

Let's look at our computation of the gcd of $514$ and $24$:
\begin{align*}
514 &= 21 \cdot 24 + 10 \\
24  &= 2 \cdot 10 + 4 \\
10  &= 2 \cdot 4 + 2 \\
4   &= 2 \cdot 2 + 0
\end{align*}
Recall that the above computation means that the gcd is 2.
Note only that through backward substitution, we can rewrite
$2$ as a linear combination of $514$ and $24$.

Let's try to do this in a more organized way.
So here's our facts again:
\begin{align*}
514 &= 21 \cdot 24 + 10 \\
24  &= 2 \cdot 10 + 4 \\
10  &= 2 \cdot 4 + 2 
\end{align*}
Let me put the remainders on one side:
\begin{align*}
10 &= 514 - 21 \cdot 24 \tag{1} \\
4  &= 24  - 2 \cdot 10  \tag{2} \\
2  &= 10  - 2 \cdot 4   \tag{3} \\
\end{align*}
Note that (1) tells you that
$10$ is a linear combination of $514,24$.
(2) tells you that $4$ is a linear combination of $24,10$.
If I substitute (1) into (2), $4$ will become a linear
combination of $514,24$.
(3) says that $2$ is a linear combination of $10,4$.
But $10$ is a linear combination of $514,24$
and $4$ is a linear combination of $514,24$.
Hence $2$ is also a linear combination of $514,24$.
See it?

OK. Let's do it.
From
\begin{align*}
10 &= 514 - 21 \cdot 24 \tag{1} \\
4  &= 24  - 2 \cdot 10  \tag{2} \\
2  &= 10  - 2 \cdot 4   \tag{3} \\
\end{align*}
if I substitute (1) into (2) and (3) (i.e., rewrite $10$ as a linear
combination of $514,24$):
\begin{align*}
10 &= 514 - 21 \cdot 24 \tag{1} \\
4  &= 24  - 2 \cdot (514 - 21 \cdot 24)  \tag{2} \\
2  &= (514 - 21 \cdot 24)  - 2 \cdot 4   \tag{3} 
\end{align*}
Collecting the multiples of 514 and 24:
\begin{align*}
10 &= 514 - 21 \cdot 24 \tag{1} \\
4  &= (-2) \cdot 514 + (1 + (-2)(-21)) \cdot 24  \tag{2'} \\
2  &= (1) \cdot 514 + (-21) \cdot 24  - 2 \cdot 4   \tag{3'} 
\end{align*}
and simplifying:
\begin{align*}
10 &= 514 - 21 \cdot 24 \tag{1} \\
4  &= (-2) \cdot 514 + (43) \cdot 24  \tag{2'} \\
2  &= (1) \cdot 514 + (-21) \cdot 24  - 2 \cdot 4   \tag{3'} 
\end{align*}
Substituting (2$'$) into (3$'$):
\begin{align*}
10 &= 514 - 21 \cdot 24 \tag{1} \\
4  &= (-2) \cdot 514 + (43) \cdot 24  \tag{2'} \\
2  &= (1) \cdot 514 + (-21) \cdot 24  - 2 \cdot ((-2) \cdot 514 + (43) \cdot 24)  \tag{3'} 
\end{align*}
Tidying up:
\begin{align*}
10 &= 514 - 21 \cdot 24 \tag{1} \\
4  &= (-2) \cdot 514 + (43) \cdot 24  \tag{2'} \\
2  &= (1 - 2(-2)) \cdot 514 + (-21 - 2(43)) \cdot 24 \tag{3''} 
\end{align*}
Simplifying:
\begin{align*}
10 &= 514 - 21 \cdot 24 \tag{1} \\
4  &= (-2) \cdot 514 + (43) \cdot 24  \tag{2'} \\
2  &= (5) \cdot 514 + (-107) \cdot 24 \tag{3''}
\end{align*}

(It's a good idea to check after each substitution that
the equalities still hold. We all make mistakes, right?)

OK.
That's great.
It looks more organized now.
So much so that you can now easily write a program to compute
the above.


Now let's look at the general case.
Suppose instead of $514$ and $24$, I write $a$ and $b$.
The computation will look like this:
\begin{align*}
a   &= q_1 \cdot b   + r_1 \\
b   &= q_2 \cdot r_1 + r_2 \\
r_1 &= q_3 \cdot r_2 + r_3 \\
r_2 &= q_4 \cdot r_3 + 0
\end{align*}
To make things even more regular and uniform, let me rewrite it this way:
\begin{align*}
r_0 &= q_1 \cdot r_1 + r_2 \\
r_1 &= q_2 \cdot r_2 + r_3 \\
r_2 &= q_3 \cdot r_3 + r_4 \\
r_3 &= q_4 \cdot r_4 + 0
\end{align*}
A lot nicer, right?
Let me write it this way with the remainder term on the lefts:
\begin{align*}
r_2 &= (1) \cdot r_0 + (-q_1) \cdot r_1 \\
r_3 &= (1) \cdot r_1 + (-q_2) \cdot r_2 \\
r_4 &= (1) \cdot r_2 + (-q_3) \cdot r_3 
\end{align*}
(Remember that $r_4$ is the gcd ... $r_0 = 514, r_1 = 24$ ... right?)
Organized this way, I have the gcd on one side of the equation.
Now if I substitute the first equation into the second I get
\begin{align*}
r_2 &= (1) \cdot r_0 + (-q_1) \cdot r_1 \text{ ... USED }\\
r_3 &= (1) \cdot r_1 + (-q_2) \cdot ((1) \cdot r_0 + (-q_1) \cdot r_1) \\
r_4 &= (1) \cdot r_2 + (-q_3) \cdot r_3 
\end{align*}
i.e.,
\begin{align*}
r_2 &= (1) \cdot r_0 + (-q_1) \cdot r_1 \text{ ... USED }\\
r_3 &= (-q_2) \cdot r_0 + (1 + q_1q_2) \cdot r_1 \\
r_4 &= (1) \cdot r_2 + (-q_3) \cdot r_3  
\end{align*}
Note that I can't throw away the first equation yet!
I need to keep $r_2$ around since it appears in the third equation!
So when can I throw the first equation away?
Look at the general case.
Suppose we have
\begin{align*}
r_2 &= (1) \cdot r_0 + (-q_1) \cdot r_1  \\
r_3 &= (1) \cdot r_1 + (-q_2) \cdot r_2  \\
r_4 &= (1) \cdot r_2 + (-q_3) \cdot r_3  \\
r_5 &= (1) \cdot r_3 + (-q_4) \cdot r_4  \\
r_6 &= (1) \cdot r_4 + (-q_5) \cdot r_5  \\
...
\end{align*}
Aha! $r_2$ is used only in the next \textit{two} equations.

Suppose we are at equation 3:
\begin{align*}
r_3 &= c_1 \cdot r_0 + d_1 \cdot r_1  \\
r_4 &= c_2 \cdot r_0 + d_2 \cdot r_1 
\end{align*}
We have to compute the next equation:
This requires $r_3, r_4$.
Then we have
\[
r_5 = (1) \cdot r_3 + (-q_4) \cdot r_4
\]
where
\[
q_4 = \floor{r_3/r_4}, \,\,\,\,\, r_5 = r_3 - q_4r_4
\]
Altogether we have
\begin{align*}
r_3 &= c_1 \cdot r_0 + d_1 \cdot r_1     \\
r_4 &= c_2 \cdot r_0 + d_2 \cdot r_1     \\
r_5 &= (1) \cdot r_3 + (-q_4) \cdot r_4 
\end{align*}
The last equation becomes
\begin{align*}
r_5 &= c_1 \cdot r_0 + d_1 \cdot r_1  
+ 
(-q_4) \cdot (c_2 \cdot r_0 + d_2 \cdot r_1)
\end{align*}
i.e.
\begin{align*}
r_5 &= (c_1 - q_4 c_2) \cdot r_0 + (d_1 - q_4 d_2) \cdot r_1 
\end{align*}

Let me repeat that in a slightly more general context.
If we have
\begin{align*}
r_3 &= c_1 \cdot r_0 + d_1 \cdot r_1  \\
r_4 &= c_2 \cdot r_0 + d_2 \cdot r_1 
\end{align*}
then we get (throwing away the first equation):
\begin{align*}
r_4 &=c_2 \cdot r_0 + d_2 \cdot r_1 \\
r_5 &= (c_1 - q_4 c_2) \cdot r_0 + (d_1 - q_4 d_2) \cdot r_1 
\end{align*}
To put it in terms of numbers instead of equations this is what we get:
If we have
\[
c_1, d_1, c_2, d_2, r_3, r_4
\]
then we get
\[
c_2, d_2, 
c_1 - \floor{r_3/r_4} c_2, 
d_1 - \floor{r_3/r_4}d_2, 
r_4, r_3 - \floor{r_3/r_4}r_4
\]
In general, if we have
\[
c, d, c', d', r, r'
\]
then we get
\[
c', d', 
c - \floor{r/r'} c', d - \floor{r/r'}d', 
r', r - \floor{r/r'}r'
\]

Of course since we start off with $r_0, r_1$ (i.e. what we call $a$ and $b$
above), we have
\begin{align*}
r_0 &= 1 \cdot r_0 + 0 \cdot r_1 \\
r_1 &= 0 \cdot r_0 + 1 \cdot r_1 \\
\end{align*}
i.e., you would start off with
\[
c=1, d=0, c'=0, d'=1, r=r_0, r'=r_1
\]
Let's check this algorithm with our $r_0 = 514, r_1 = 24$.

STEP 1: The initial numbers are
\[
c=1, d=0, c'=0, d'=1, r=514, r'=24
\]
Again this corresponds to
\begin{align*}
r_3 &= 1 \cdot 514 + 0 \cdot 24  \\
r_4 &= 0 \cdot 514 + 1 \cdot 24 \\
\end{align*}

STEP 2: 
The new numbers (6 of them) are
\begin{align*}
 c' &=0 \\
 d' &=1 \\
 c - \floor{r/r'} c' &= 1 - \floor{514/24} 0 = 1\\
 d - \floor{r/r'}d' &= 0 - \floor{514/24}1 = 0 - 21 = -21 \\
 r' &= 24 \\ 
 r - \floor{r/r'}r' &= 514 - \floor{514/24}24 = 514 - 504 = 10
\end{align*}
So the new numbers (we reset the variables in the algorithm):
\[
c=0, d=1, c'=1, d'=-21, r=24, r'= 10
\]
These corresponds to the data on the second and third line of the following:
\begin{align*}
514 &= 1 \cdot 514 + 0 \cdot 24      \\
24  &= 0 \cdot 514 + 1 \cdot 24      \\
10  &= 1 \cdot 514 + (-21) \cdot 24 
\end{align*}

STEP 3: From the 6 numbers from STEP 2 we get
\begin{align*}
 c' &= 1 \\
 d' &= -21 \\
 c - \floor{r/r'} c' &= 0 - \floor{24/10} 1 = -2 \\
 d - \floor{r/r'}d'  &= 1 - \floor{24/10} (-21) = 1 + 42 = 43 \\
 r' &= 10 \\ 
 r - \floor{r/r'}r' &= 24 - \floor{24/10}10 = 24 - 20 = 4
\end{align*}
So the new numbers (we reset the variables in the algorithm):
\[
c=1, d=-21, c'=-2, d'= 43, r=10, r'= 4
\]
These corresponds to the data on the third and fourth line of the following:
\begin{align*}
514 &= 1 \cdot 514 + 0 \cdot 24     \\
24  &= 0 \cdot 514 + 1 \cdot 24     \\
10  &= 1 \cdot 514 + (-21) \cdot 24 \\
4   &= (-2) \cdot 514 + 43 \cdot 24 
\end{align*}

STEP 4: From the 6 numbers from STEP 3 we get
\begin{align*}
 c' &= -2 \\
 d' &= 43 \\
 c - \floor{r/r'} c' &= 1 - \floor{10/4} (-2) = 1 + 4 = 5 \\
 d - \floor{r/r'}d'  &= -21 - \floor{10/4} (43) = -21 - 86 = -107 \\
 r' &= 4 \\ 
 r - \floor{r/r'}r' &= 10 - \floor{10/4} 4 = 10 - 8 = 2
\end{align*}
So the new numbers (we reset the variables in the algorithm):
\[
c = -2, d = 43, c' = 5, d'= -107, r=4, r'= 2
\]
These corresponds to the data on the fourth and fifth line of the following:
\begin{align*}
514 &= 1 \cdot 514 + 0 \cdot 24      \\
24  &= 0 \cdot 514 + 1 \cdot 24      \\
10  &= 1 \cdot 514 + (-21) \cdot 24  \\
4   &= (-2) \cdot 514 + 43 \cdot 24  \\
2   &= 5 \cdot 514 + (-107) \cdot 24 
\end{align*}

STEP 5: From the 6 numbers from STEP 4 we get
\begin{align*}
 c' &= 5 \\
 d' &= -107 \\
 c - \floor{r/r'} c' &= -2 - \floor{4/2} 5 = -2 - 10 = -12 \\
 d - \floor{r/r'} d'  &= 43 - \floor{4/2} (-107) = 43 + 214 = 257 \\
 r' &= 2 \\ 
 r - \floor{r/r'}r' &= 4 - \floor{4/2} 2 = 4 - 4 = 0
\end{align*}
So the new numbers (we reset the variables in the algorithm):
\[
c = 5, d = -107, c' = -12, d'= 257, r=2, r'= 0
\]
These corresponds to the data on the fifth and sixth line of the following:
\begin{align*}
514 &= 1 \cdot 514 + 0 \cdot 24       \\
24  &= 0 \cdot 514 + 1 \cdot 24       \\
10  &= 1 \cdot 514 + (-21) \cdot 24   \\
4   &= (-2) \cdot 514 + 43 \cdot 24   \\
2   &= 5 \cdot 514 + (-107) \cdot 24  \\
0   &= (-12) \cdot 514 + 257 \cdot 24 
\end{align*}

Of course (as before) at this point, you see that the $r'=0$.
Therefore 
\[
\gcd(514, 24) = 2
\]
and furthermore from $c=5, d=-107$, we get
\[
5 \cdot 514 + (-107) \cdot 24 = \gcd(514, 24) 
\]

Here's a Python implementation with some test code:
\begin{Verbatim}[frame=single, fontsize=\small]
ALGORITHM: EEA
INPUTS: a, b
OUTPUTS: r, c, d where r = gcd(a, b) = c*a + d*b

    a0, b0 = a, b
    d0, d = 0, 1
    c0, c = 1, 0
    q = a0 // b0
    r = a0 - q * b0
    
    while r > 0:
        d, d0 = d0 - q * d, d    
        c, c0 = c0 - q * c, c
    
        a0, b0 = b0, r
        q = a0 // b0
        r = a0 - q * b0

    r = b0
    return r, c, d
\end{Verbatim}
You can pound real hard at the Extended Euclidean Algorithm with this:

By the way, this is somewhat similar to what we call
\textit{tail recursion} (CISS445)
an extremely important technique in functional programming.
All LISP hackers and people working in high performance computing
and compilers swear by it.
You don't see recursion in the above code, but you can 
replace the while-loop with recursion and if
you have a compiler/interpreter that can perform 
true tail recursion, then it would run exactly like the above algorithm.

\begin{ex}
  Leetcode 365:
  \url{https://leetcode.com/problems/water-and-jug-problem/description/}
  and the Die Hard 3 problem
  \url{https://www.math.tamu.edu/~dallen/hollywood/diehard/diehard.htm}.
  You are given two jugs with capacities \verb!jug1Capacity! and
  \verb!jug2Capacity! liters.
  There is an infinite amount of water supply available.
  Determine whether it is possible to measure exactly \verb!targetCapacity!
  liter using these two jugs.

  If \verb!targetCapacity! liters of water are measurable,
  you must have \verb!targetCapacity! liters of water contained within one or
  both buckets by the end.

  Operations allowed:
  \begin{enumerate}[nosep]
    \item Fill any of the jugs with water.
    \item Empty any of the jugs.
    \item Pour water from one jug into another till the other jug is completely
      full, or the first jug itself is empty.
  \end{enumerate}
  
\end{ex}

You'll see that there are times when you're only interested in 
the value of $x$ and not $y$ (or $y$ and not $x$ -- the above is symmetric
about $x$ and $y$).
Do you notice $x$ comes from $c$?
If you analyze the above algorithm, you see immediately that $c$
is computed from $c'$ and $c'$ is computed from $c,c',q$, 
$q$ is computed from $r, r'$, $r$ is computed from $r'$,
and finally (phew!) $r'$ is computed from $r, q, r'$.
Therefore if you're interested in $c$, you don't need to compute $d$ 
or $d'$.
So you can change the EEA to this:



\begin{Verbatim}[frame=single, fontsize=\small]
ALGORTHM: EEA2 (sort of EEA ... without the d, d0)
INPUTS: a, b
OUTPUTS: r, c where r = gcd(a, b) = c*a + d*b for some d

    a0, b0 = a, b
    c0, c = 1, 0
    q = a0 // b0
    r = a0 - q * b0

    while r > 0:   
        c, c0 = c0 - q * c, c
    
        a0, b0 = b0, r
        q = a0 // b0
        r = a0 - q * b0

    r = b0
    return r, c
\end{Verbatim}

Later you'll see why we compute only $c$.
It's not that I have something against $d$.

\newpage
\begin{ex}
  Compute the following gcd and the B\'ezout's coefficients of the
  given numbers by following the Extended Euclidean Algorithm.
  \begin{enumerate}[nosep]
  \item $\gcd(0, 10)$
  \item $\gcd(10, 0)$
  \item $\gcd(10, 1)$
  \item $\gcd(10, 10)$
  \item $\gcd(107, 5)$
  \item $\gcd(107, 26)$
  \item $\gcd(84, 333)$
  \item $\gcd(F_{10}, F_{11})$ where $F_n$ is the $n$--th Fibonacci number.
    (Recall: $F_0 = 0, F_1 = 1, F_{n + 2} = F_{n + 1} + F_n$.)
  \item $\gcd(ab, b)$
  \item $\gcd(a, a + 1)$
  \item $\gcd(ab + a, b)$ where $0 < a < b$. Go as far as you can.
  \item $\gcd(a(a+1) + a, (a+1))$ where $0 < a < b$. Go as far as you can.
  \end{enumerate}
\end{ex}

\begin{ex}
  Prove that if $a \mid c$, $b \mid c$, and $\gcd(a,b) = 1$, then
  $ab \mid c$.
\end{ex}

\begin{ex}
  Prove that if $a \mid c$, $b \mid c$, then
  \[
  \frac{ab}{\gcd(a,b)} \mid c
  \]
\end{ex}


\begin{ex}
  Leetcode 920.
  \url{https://leetcode.com/problems/number-of-music-playlists}
  Your music player contains n different songs. You want to listen to goal songs (not necessarily different) during your trip. To avoid boredom, you will create a playlist so that:

Every song is played at least once.

A song can only be played again only if k other songs have been played.

Given \verb!n!, \verb!goal!, and \verb!k!, return the number of possible playlists that you can create. Since the answer can be very large, return it modulo $109 + 7$.
\end{ex}
