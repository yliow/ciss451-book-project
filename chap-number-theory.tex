\chapter{Basic number theory}

\boxpar{
\textsc{Suggestions}.
For this chapter, state the basic axioms and properties/theorems of $\Z$.
Provide proofs. 
But remember that most of the properties/theorems can be generlized
to properties/theorems for rings.
It's still a good idea to prove the facts for $\Z$ since $\Z$ is not
as abstract as general rings and will prepare you for the general results.
}

\section{Axioms of $\Z$}

We will assume that $(\Z, +, \cdot, 0, 1)$ satisfies the following
axioms.
\begin{enumerate}[nosep]
  \li \textsc{Properties of $+$:}
  \begin{enumerate}[nosep]
    \li Closure: If $x, y \in \Z$, then $x + y \in \Z$.
    \li Associativity: If $x, y, z \in \Z$, then $(x + y) + z = x + (y + z)$.
    \li Inverse: If $x \in \Z$, then there is some $y$ such that $x + y = 0 = y + x$.
      The $y$ in the above is an additive inverse of $x$.
    \li Neutrality: If $x \in \Z$, then $0 + x = x = x + 0$.
    \li Commutativity: If $x, y \in \Z$, then $x + y = y + x$.
  \end{enumerate}
  (Memory aid for the first four: CAIN.)
  \li \textsc{Properties of $\cdot$:}
  \begin{enumerate}[nosep]
  \li Closure: If $x, y \in \Z$, then $x \cdot y \in \Z$.
  \li Associativity: If $x, y, z \in \Z$, then $(x \cdot y) \cdot z = x \cdot (y \cdot z)$.
  \li Neutrality: If $x \in \Z$, then $1 \cdot x = x = x + 1$.
  \li Commutativity: If $x, y \in \Z$, then $x \cdot y = y \cdot x$.
  \end{enumerate}
  It is common to write $xy$ instead of $x \cdot y$.

  \li \textsc{Distributivity}:
  If $x, y, z \in \Z$, then
  $x \cdot (y + z) = x \cdot y + x \cdot z$
  and
  $(y + z) \cdot x = y \cdot x + z \cdot x$
    
\end{enumerate}

A set $R$ with operations $+, \cdot$ and elements $0_R, 1_R$ satisfying
the above properties the above is called a commutative ring.
This is an important generalization because there are many
very useful commutative rings and we want to prove results about
commutative rings so that these results can be applied to all
commutative rings, including but not restricted to $\Z$.
And if the commutativity of multiplication is left out, then we have the
concept of a ring; for emphasize these are called non-commutative rings.
This is also an important concept since $n \times n$ matrices with $\R$
entries, $\operatorname{M}_{n \times n}(\R)$, form a non-commutative ring.
In fact, more generally,
the set of $n\times n$ matrices with entries in a commutative ring $R$,
$\operatorname{M}_{n \times n}(R)$,
is itself a non-commutative ring.
We will return to the concept of commutative and non-commutative rings later.

The next property of $\Z$, integrality,
is very special and does not apply to many
communtative rings and is therefore
left out of the definition of commutative ring:
\begin{enumerate}[nosep]
  \li \textsc{Integrality}:
  If $x, y \in \Z$, then
  $xy = 0 \implies x = 0$ or $y = 0$.
\end{enumerate}
Another property of $\Z$ that we will assume is
\begin{enumerate}[nosep]
  \li \textsc{Nontriviality}:
  $0 \neq 1$
\end{enumerate}
This axiom of $\Z$ is extremely simple, but cannot be deduced from
the previous axioms.

The above forms the algebraic properties of $\Z$, i.e., properties involving
addition and multiplication.

It is actually possible to first define axioms for
$\N = \{0, 1, 2, ...\}$ and then define $\Z$ in terms of $\N$.
We will not do that except to mention that the
axioms for $\N$ are called the
\href{https://en.wikipedia.org/wiki/Giuseppe_Peano}{Peano}-\href{https://en.wikipedia.org/wiki/Richard_Dedekind}{Dedekind} axioms
and that one very important Peano-Dedekind axiom of $\N$ is the
\begin{enumerate}[nosep]
  \li \textsc{Well-ordering principle (WOP)} for $\N$:
  If $X$ is a nonempty subset of $\N$, then $X$ contains a minimum element,
  i.e., there is some $m \in X$ such that
  \[
  m \leq x
  \]
  for all $x \in X$.
\end{enumerate}
Without going into details, it can be shown that for $\N$, the WOP
is equivalent to each of the following axioms:
\begin{enumerate}[nosep]
  \li \textsc{Weak Mathematical Induction} for $\N$:
  Let $X$ be a subset of $\N$ satisyfing the following two conditions:
  \begin{enumerate}[nosep]
    \li $0 \in X$ and
    \li Let $n \in \N$. If $n \in X$, then $n + 1 \in X$.
  \end{enumerate}
  Then $X = \N$.
\end{enumerate}
and
\begin{enumerate}[nosep]
  \li \textsc{Strong Mathematical Induction} for $\N$:
  Let $X$ be a subset of $\N$ satisyfing the following two conditions:
  \begin{enumerate}[nosep]
    \li $0 \in X$ and
    \li Let $n \in \N$. If $k \in X$ for all $0 \leq k \leq n$,
    then $n + 1 \in X$.
  \end{enumerate}
  Then $X = \N$.
\end{enumerate}
  
In the above two induction axioms, if we write $X = \{n \mid P(n)\}$ where
$P(n)$ is a propositional formula,
then the induction axioms can be rewritten in the
following way:

\begin{enumerate}[nosep]
  \li \textsc{Weak Mathematical Induction}:
  Let $P(n)$ be a proposition for $n \in \N$ satisyfing the following two
  conditions:
  \begin{enumerate}[nosep]
    \li $P(0)$ is true and
    \li Let $n \in \N$. If $P(n)$ is true, then $P(n+1)$ is true.
  \end{enumerate}
  Then $P(n)$ is true for all $n \in \N$.
\end{enumerate}
and
\begin{enumerate}[nosep]
  \li \textsc{Strong Mathematical Induction}:
  Let $P(n)$ be a proposition for $n \in \N$ satisyfing the following two
  conditions:
  \begin{enumerate}[nosep]
    \li $P(0)$ is true and
    \li Let $n \in \N$. If $k \in X$ for all $0 \leq k \leq n$,
    then $n + 1 \in X$.
    \li Let $n \in \N$. If $P(k)$ is true for $0 \leq k \leq n$,
    then $P(n+1)$ is true.
  \end{enumerate}
  Then $P(n)$ is true for all $n \in \N$.
\end{enumerate}

The above are the algebraic axioms of $\Z$.
There's also the order relation of $\Z$ which is used in WOP
and the two induction principles.
I will formalize the axioms of the order relation later.
For now one can assume that the order relation is defined as follows:
If $x \in \Z$, then
\[
x < y
\]
if there is some $z \in \N$ such that
\[
x + z = y
\]
There is one more axiom of $\Z$ that is related to the
\lq\lq topology" of $\Z$ and uses the order relation:

\begin{enumerate}[nosep]
  \li \textsc{Topology}:
  Given any $x \in \Z$, there is no $y \in \Z$ such that
  \[
  x < y < x + 1
  \]
\end{enumerate}

You can think of topology of a set as study of \lq\lq closeness" of values
in that set.
For $\Q$, given any two distinct
rational values $x < y$, there is also some $z \in \Q$ such that
$x < z < y$.
This is the same for $\R$.
Therefore the topology of $\Z$ is very different from the topology
of $\Q$ and $\R$ because there are \lq\lq holes" in $\Z$ where
there are no $\Z$ values.
$\Z$ has what is called a discrete topology.

The above assume the existence of an order relation on $\Z$, i.e., $<$.
We have to include the following axioms of $<$ on $\Z$.
There is a set $\Z^+$ such that the following holds:
\begin{enumerate}[nosep]
  \li \textsc{Trichotomy}: If $x \in \Z$, then exactly one of the following holds:
  $-x \in \Z^+$, $x = 0$, $x \in \Z^+$.
  \li \textsc{Closure of $+$}: If $x,y \in \Z^+$, then $x + y \in \Z^+$.
  \li \textsc{Closure of $\cdot$}: If $x,y \in \Z^+$, then $x \cdot y \in \Z^+$.
\end{enumerate}
We then define $<$ as follows: If $x, y \in \Z$, then we write $x < y$ if
\[
y - x \in \Z^+
\]
Since $<$ is defined, we can define $x \leq y$ to mean \lq\lq either $x < y$
or $x = y$".
The above order relation is expressed abstractly without refering to the fact that
$\Z^+$ is $\{1, 2, 3, ...\}$, i.e., the set of positive integers.
In fact, you can prove $\Z^+ = \{1, 2, 3, ...\}$ from the above axioms --
see exercises below.

\begin{prop}
  The additive inverse for $x$ is unique.
  In other words, if $y, y'$ satisfies
  \begin{align*}
    x + y &= 0 = y + x \\
    x + y' &= 0 = y' + x \\
  \end{align*}
  Then $y = y'$.
\end{prop}

Since the additive inverse of $x$ is unique, we can choose to write
the additive inverse of $x$ in terms of $x$.
This is usually written $-x$.
We define the operator $-$ in terms of the additive inverse:

\begin{defn}
  Let $x,y \in \Z$. We define the subtraction operator as
  \[
  x - y = x + (-y)
  \]
\end{defn}

Note that every $x$ in $\Z$ as an additive inverse, but we did not
require value of $\Z$ to have multiplicative inverse:

\begin{defn}
  Let $x \in \Z$. Then $y$ is a multiplicative inverse of $x$ if
  \[
  x\cdot y = 1 = y\cdot x
  \]
  We say that $x$ is a \defone{unit} if $x$ has a multiplicative inverse.
  We can also say that $x$ is \defone{invertible}.
\end{defn}

Intuitively, you know that the only values of $\Z$ with multiplicative inverses
are $1$ and $-1$.

\begin{prop}
  Let $x \in \Z$.
  If $x$ is a unit, then the multiplicative inverse of $x$ is unique.
  In other words if $y,y'$ satisfies
  \begin{align*}
    x  y &= 1 = y x \\
    x  y' &= 1 = y' x \\
  \end{align*}
  then $y = y'$.
\end{prop}

\begin{defn}
  The multiplicative inverse of $x$, it is exists, is is denoted by $x^{-1}$.
\end{defn}

\begin{prop}
  Cancellation law for addition.
  Let $x, y, z \in \Z$.
  \begin{enumerate}[nosep,label=\textnormal{(\alph*)}]
    \item If $x + z = y + z$, then $x = y$.
    \item If $z + x = z + y$, then $x = y$.
  \end{enumerate}
\end{prop}

\begin{prop}
  Let $x \in \Z$.
  \begin{enumerate}[nosep,label=\textnormal{(\alph*)}]
  \item $0x = 0 = x0$
  \item $-0 = 0$
  \item $x - 0 = x = 0 - x$.
  \end{enumerate}
\end{prop}
\proof
(a) We will first prove $0x = 0$:
\begin{align*}
  0x &= (0 + 0)x & & \text{by Neutrality of $+$} \\
     &= 0x + 0x  & & \text{by Distributivity} \tag{1}
\end{align*}
Since $0x \in \Z$ by Closure of $\cdot$, there is
exists some $y \in \Z$ which is an additive inverse of $0x$, i.e.,
\[
0x + y = 0 = y + 0x \tag{2}
\]
From (1),
\begin{align*}
  y + 0x &= y + (0x + 0x)  \\
  0      &= y + (0x + 0x) & & \text{ by (2)} \\
  0      &= (y + 0x) + 0x & & \text{ by Associativity of $+$} \\
  0      &= 0 + 0x        & & \text{ by (2)} \\
  0      &= 0x            & & \text{ by Neutrality of $+$}
\end{align*}
To prove $0 = x0$, from above
\begin{align*}
  0 &= 0x \\
    &= x0 & & \text{ by Commutativity of $\cdot$}
\end{align*}

(b) TODO

(c) TODO
\qed

\begin{prop}
  Let $x, y \in \Z$.
  \begin{enumerate}[nosep,label=\textnormal{(\alph*)}]
  \item $-(-1) = 1$
  \item $-(-x) = x$
  \item $x(-1) = -x = (-1)x$
  \item $(-1)(-1) = 1$
  \item $(-x)(-y) = xy$
  \end{enumerate}
\end{prop}

\begin{prop}
  Cancellation law for multiplication.
  Let $x, y, z \in \Z$.
  \begin{enumerate}[nosep,label=\textnormal{(\alph*)}]
    \item If $xz = yz$ and $z \neq 0$, then $x = y$.
    \item If $zx = zy$ and $z \neq 0$, then $x = y$.
  \end{enumerate}
\end{prop}

For convenience, I will write $x^2 = xx$ and in general
\[
x^n =
\begin{cases}
  1       &\text{if } n = 0 \\
  x^{n-1}x &\text{if } n > 0 \\
\end{cases}
\]
If $x$ has a multiplicative inverse, i.e., if $x^{-1}$ exists,
then, for $n \geq 0$, I will define
\[
x^{-n} = \left( x^{-1} \right)^n
\]

\begin{prop}
  Let $x \in \Z$.
  Then $[n]x = n \cdot x$.
\end{prop}

Note that $nx$ has two meanings: $nx$ can be the multiplication of $n$
and $x$ and it can also be $x + \cdots + x$ with $n$ number of $x$.
Of course you would expect them to be the same.
For now define
\[
  [n]x =
  \begin{cases}
    0       &\text{if } n = 0 \\
    [n-1]x + x &\text{if } n > 0 \\
  \end{cases}
\]
and if $n$ is negative, we define
\[
[n]x = -([-n]x)
\]

\newpage
\section{Divisibility}
\begin{defn}
  Let $m, n \in \Z$.
  Then we say that $m$ \defterm{divides} $n$, and we write $m \mid n$, if
  there is some $x \in \Z$ such that $mx = n$, i.e.,
  \[
    \exists x \in \Z \cdot [mx = n]
  \]
\end{defn}


\begin{prop}
  Let $a \in \Z$.
  \begin{enumerate}[nosep,label=\textnormal{(\alph*)}]
  \item Then $1 \mid a$.
  \item If $a \neq 0$, then $a \mid 0$.
  \item (Reflexive) If $a \neq 0$, then $a \mid a$.
  \item If $a \mid b$ and $b \mid a$, then $a = \pm b$.  
  \item (Transitive) If $a \mid b$ and $b \mid c$, then $a \mid c$.
  \item If $a \mid b$, then $a \mid bc$.
  \item If $a \mid b$, $a \mid c$, then $a \mid b + c$.
  \item (Linearity)
    If $a \mid b$, $a \mid c$, then $a \mid bx + cy$ for $x,y \in \Z$.  
  \end{enumerate}
\end{prop}


\newpage
\section{Congruences}

\begin{defn}
  Let $a, b \in \Z$ and $N \in \Z$ with $N > 0$.
  Then $a$ is congruent to $b$ mod $N$ and we write
  \[
  a \equiv b \pmod{N}
  \]
  if $N \mid a - b$.
\end{defn}


\begin{prop}
  Let $a,b,c,a',b' \in \Z$.
  \begin{enumerate}[nosep,label=\textnormal{(\alph*)}]
  \item (Reflexivity) $a \equiv a \pmod{N}$
  \item (Symmetry) If $a \equiv b \pmod{N}$, then $b \equiv a \pmod{N}$
  \item (Transitivity)
    If $a \equiv b, b \equiv c \pmod{N}$, then
    $a \equiv c \pmod{N}$
  \item
    If
    $a \equiv b$,
    $a' \equiv b' \pmod{N}$,
    then
    $a + a' \equiv b + b' \pmod{N}$.
  \item
    If
    $a \equiv b$,
    $a' \equiv b' \pmod{N}$,
    then
    $a a' \equiv b b' \pmod{N}$.
  \end{enumerate}
\end{prop}


\begin{prop}
  Let $a, N \in \Z$ with $N > 0$.
  Let $q, r \in \Z$ such that
  \[
  a = Nq + r, \,\,\, 0 \leq r < N
  \]
  Then $a \equiv r \pmod{N}$.
\end{prop}
