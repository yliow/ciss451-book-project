\chapter{Basic number theory}

\boxpar{
\textsc{Suggestions}.
For this chapter, state the basic axioms and properties/theorems of $\Z$.
Provide proofs. 
But remember that most of the properties/theorems can be generlized
to properties/theorems for rings.
It's still a good idea to prove the facts for $\Z$ since $\Z$ is not
as abstract as general rings and will prepare you for the general results.
}

\section{Axioms of $\Z$}

We will assume that $(\Z, +, \cdot, 0, 1)$ satisfies the following
axioms.
\begin{enumerate}[nosep]
  \li \textsc{Properties of $+$:}
  \begin{enumerate}[nosep]
    \li Closure: If $x, y \in \Z$, then $x + y \in \Z$.
    \li Associativity: If $x, y, z \in \Z$, then $(x + y) + z = x + (y + z)$.
    \li Inverse: If $x \in \Z$, then there is some $y$ such that $x + y = 0 = y + x$.
      The $y$ in the above is an additive inverse of $x$.
    \li Neutrality: If $x \in \Z$, then $0 + x = x = x + 0$.
    \li Commutativity: If $x, y \in \Z$, then $x + y = y + x$.
  \end{enumerate}
  (Memory aid for the first four: CAIN.)
  \li \textsc{Properties of $\cdot$:}
  \begin{enumerate}[nosep]
  \li Closure: If $x, y \in \Z$, then $x \cdot y \in \Z$.
  \li Associativity: If $x, y, z \in \Z$, then $(x \cdot y) \cdot z = x \cdot (y \cdot z)$.
  \li Neutrality: If $x \in \Z$, then $1 \cdot x = x = x + 1$.
  \li Commutativity: If $x, y \in \Z$, then $x \cdot y = y \cdot x$.
  \end{enumerate}
  It is common to write $xy$ instead of $x \cdot y$.

  \li \textsc{Distributivity}:
  If $x, y, z \in \Z$, then
  $x \cdot (y + z) = x \cdot y + x \cdot z$
  and
  $(y + z) \cdot x = y \cdot x + z \cdot x$
    
\end{enumerate}

A set $R$ with operations $+, \cdot$ and elements $0_R, 1_R$ satisfying
the above properties the above is called a commutative ring.
This is an important generalization because there are many
very useful commutative rings and we want to prove results about
commutative rings so that these results can be applied to all
commutative rings, including but not restricted to $\Z$.
And if the commutativity of multiplication is left out, then we have the
concept of a ring; for emphasize these are called non-commutative rings.
This is also an important concept since $n \times n$ matrices with $\R$
entries, $\operatorname{M}_{n \times n}(\R)$, form a non-commutative ring.
In fact, more generally,
the set of $n\times n$ matrices with entries in a commutative ring $R$,
$\operatorname{M}_{n \times n}(R)$,
is itself a non-commutative ring.
We will return to the concept of commutative and non-commutative rings later.

The next property of $\Z$, integrality,
is very special and does not apply to many
communtative rings and is therefore
left out of the definition of commutative ring:
\begin{enumerate}[nosep]
  \li \textsc{Integrality}:
  If $x, y \in \Z$, then
  $xy = 0 \implies x = 0$ or $y = 0$.
\end{enumerate}
Another property of $\Z$ that we will assume is
\begin{enumerate}[nosep]
  \li \textsc{Nontriviality}:
  $0 \neq 1$
\end{enumerate}
This axiom of $\Z$ is extremely simple, but cannot be deduced from
the previous axioms.

The above forms the algebraic properties of $\Z$, i.e., properties involving
addition and multiplication.

It is actually possible to first define axioms for
$\N = \{0, 1, 2, ...\}$ and then define $\Z$ in terms of $\N$.
We will not do that except to mention that the
axioms for $\N$ are called the
\href{https://en.wikipedia.org/wiki/Giuseppe_Peano}{Peano}-\href{https://en.wikipedia.org/wiki/Richard_Dedekind}{Dedekind} axioms
and that one very important Peano-Dedekind axiom of $\N$ is the
\begin{enumerate}[nosep]
  \li \textsc{Well-ordering principle (WOP)} for $\N$:
  If $X$ is a nonempty subset of $\N$, then $X$ contains a minimum element,
  i.e., there is some $m \in X$ such that
  \[
  m \leq x
  \]
  for all $x \in X$.
\end{enumerate}
Without going into details, it can be shown that for $\N$, the WOP
is equivalent to each of the following axioms:
\begin{enumerate}[nosep]
  \li \textsc{Weak Mathematical Induction} for $\N$:
  Let $X$ be a subset of $\N$ satisyfing the following two conditions:
  \begin{enumerate}[nosep]
    \li $0 \in X$ and
    \li Let $n \in \N$. If $n \in X$, then $n + 1 \in X$.
  \end{enumerate}
  Then $X = \N$.
\end{enumerate}
and
\begin{enumerate}[nosep]
  \li \textsc{Strong Mathematical Induction} for $\N$:
  Let $X$ be a subset of $\N$ satisyfing the following two conditions:
  \begin{enumerate}[nosep]
    \li $0 \in X$ and
    \li Let $n \in \N$. If $k \in X$ for all $0 \leq k \leq n$,
    then $n + 1 \in X$.
  \end{enumerate}
  Then $X = \N$.
\end{enumerate}
  
In the above two induction axioms, if we write $X = \{n \mid P(n)\}$ where
$P(n)$ is a propositional formula,
then the induction axioms can be rewritten in the
following way:

\begin{enumerate}[nosep]
  \li \textsc{Weak Mathematical Induction}:
  Let $P(n)$ be a proposition for $n \in \N$ satisyfing the following two
  conditions:
  \begin{enumerate}[nosep]
    \li $P(0)$ is true and
    \li Let $n \in \N$. If $P(n)$ is true, then $P(n+1)$ is true.
  \end{enumerate}
  Then $P(n)$ is true for all $n \in \N$.
\end{enumerate}
and
\begin{enumerate}[nosep]
  \li \textsc{Strong Mathematical Induction}:
  Let $P(n)$ be a proposition for $n \in \N$ satisyfing the following two
  conditions:
  \begin{enumerate}[nosep]
    \li $P(0)$ is true and
    \li Let $n \in \N$. If $k \in X$ for all $0 \leq k \leq n$,
    then $n + 1 \in X$.
    \li Let $n \in \N$. If $P(k)$ is true for $0 \leq k \leq n$,
    then $P(n+1)$ is true.
  \end{enumerate}
  Then $P(n)$ is true for all $n \in \N$.
\end{enumerate}

The above are the algebraic axioms of $\Z$.
There's also the order relation of $\Z$ which is used in WOP
and the two induction principles.
I will formalize the axioms of the order relation later.
For now one can assume that the order relation is defined as follows:
If $x \in \Z$, then
\[
x < y
\]
if there is some $z \in \N$ such that
\[
x + z = y
\]
There is one more axiom of $\Z$ that is related to the
\lq\lq topology" of $\Z$ and uses the order relation:

\begin{enumerate}[nosep]
  \li \textsc{Topology}:
  Given any $x \in \Z$, there is no $y \in \Z$ such that
  \[
  x < y < x + 1
  \]
\end{enumerate}

You can think of topology of a set as study of \lq\lq closeness" of values
in that set.
For $\Q$, given any two distinct
rational values $x < y$, there is also some $z \in \Q$ such that
$x < z < y$.
This is the same for $\R$.
Therefore the topology of $\Z$ is very different from the topology
of $\Q$ and $\R$ because there are \lq\lq holes" in $\Z$ where
there are no $\Z$ values.
$\Z$ has what is called a discrete topology.

The above assume the existence of an order relation on $\Z$, i.e., $<$.
We have to include the following axioms of $<$ on $\Z$.
There is a set $\Z^+$ such that the following holds:
\begin{enumerate}[nosep]
  \li \textsc{Trichotomy}: If $x \in \Z$, then exactly one of the following holds:
  $-x \in \Z^+$, $x = 0$, $x \in \Z^+$.
  \li \textsc{Closure of $+$}: If $x,y \in \Z^+$, then $x + y \in \Z^+$.
  \li \textsc{Closure of $\cdot$}: If $x,y \in \Z^+$, then $x \cdot y \in \Z^+$.
\end{enumerate}
We then define $<$ as follows: If $x, y \in \Z$, then we write $x < y$ if
\[
y - x \in \Z^+
\]
Since $<$ is defined, we can define $x \leq y$ to mean \lq\lq either $x < y$
or $x = y$".
The above order relation is expressed abstractly without refering to the fact that
$\Z^+$ is $\{1, 2, 3, ...\}$, i.e., the set of positive integers.
In fact, you can prove $\Z^+ = \{1, 2, 3, ...\}$ from the above axioms --
see exercises below.

\begin{prop}
  The additive inverse for $x$ is unique.
  In other words, if $y, y'$ satisfies
  \begin{align*}
    x + y &= 0 = y + x \\
    x + y' &= 0 = y' + x \\
  \end{align*}
  Then $y = y'$.
\end{prop}
\proof
TODO

Since the additive inverse of $x$ is unique, we can choose to write
the additive inverse of $x$ in terms of $x$.
This is usually written $-x$.
We define the operator $-$ in terms of the additive inverse:

\begin{defn}
  Let $x,y \in \Z$. We define the subtraction operator as
  \[
  x - y = x + (-y)
  \]
\end{defn}

Note that every $x$ in $\Z$ as an additive inverse, but we did not
require value of $\Z$ to have multiplicative inverse:

\begin{defn}
  Let $x \in \Z$. Then $y$ is a multiplicative inverse of $x$ if
  \[
  x\cdot y = 1 = y\cdot x
  \]
  We say that $x$ is a \defone{unit} if $x$ has a multiplicative inverse.
  We can also say that $x$ is \defone{invertible}.
\end{defn}

Intuitively, you know that the only values of $\Z$ with multiplicative inverses
are $1$ and $-1$.

\begin{prop}
  Let $x \in \Z$.
  If $x$ is a unit, then the multiplicative inverse of $x$ is unique.
  In other words if $y,y'$ satisfies
  \begin{align*}
    x  y &= 1 = y x \\
    x  y' &= 1 = y' x \\
  \end{align*}
  then $y = y'$.
\end{prop}

\begin{defn}
  The multiplicative inverse of $x$, it is exists, is is denoted by $x^{-1}$.
\end{defn}

\begin{prop}
  Cancellation law for addition.
  Let $x, y, z \in \Z$.
  \begin{enumerate}[nosep,label=\textnormal{(\alph*)}]
    \item If $x + z = y + z$, then $x = y$.
    \item If $z + x = z + y$, then $x = y$.
  \end{enumerate}
\end{prop}

\begin{prop}
  Let $x \in \Z$.
  \begin{enumerate}[nosep,label=\textnormal{(\alph*)}]
  \item $0x = 0 = x0$
  \item $-0 = 0$
  \item $x - 0 = x$. \textred(NOTE CORRECTION)
  \end{enumerate}
\end{prop}
\proof
(a) We will first prove $0x = 0$:
\begin{align*}
  0x &= (0 + 0)x & & \text{by Neutrality of $+$} \\
     &= 0x + 0x  & & \text{by Distributivity} \tag{1}
\end{align*}
Since $0x \in \Z$ by Closure of $\cdot$, there is
exists some $y \in \Z$ which is an additive inverse of $0x$, i.e.,
\[
0x + y = 0 = y + 0x \tag{2}
\]
From (1),
\begin{align*}
  y + 0x &= y + (0x + 0x)  \\
  0      &= y + (0x + 0x) & & \text{ by (2)} \\
  0      &= (y + 0x) + 0x & & \text{ by Associativity of $+$} \\
  0      &= 0 + 0x        & & \text{ by (2)} \\
  0      &= 0x            & & \text{ by Neutrality of $+$}
\end{align*}
To prove $0 = x0$, from above
\begin{align*}
  0 &= 0x \\
    &= x0 & & \text{ by Commutativity of $\cdot$}
\end{align*}

(b) TODO

(c) TODO
\qed

\begin{prop}
  Let $x, y \in \Z$.
  \begin{enumerate}[nosep,label=\textnormal{(\alph*)}]
  \item $-(-1) = 1$
  \item $-(-x) = x$
  \item $x(-1) = -x = (-1)x$
  \item $(-1)(-1) = 1$
  \item $(-x)(-y) = xy$
  \end{enumerate}
\end{prop}

\begin{prop}
  Cancellation law for multiplication.
  Let $x, y, z \in \Z$.
  \begin{enumerate}[nosep,label=\textnormal{(\alph*)}]
    \item If $xz = yz$ and $z \neq 0$, then $x = y$.
    \item If $zx = zy$ and $z \neq 0$, then $x = y$.
  \end{enumerate}
\end{prop}

For convenience, I will write $x^2 = xx$ and in general
\[
x^n =
\begin{cases}
  1       &\text{if } n = 0 \\
  x^{n-1}x &\text{if } n > 0 \\
\end{cases}
\]
If $x$ has a multiplicative inverse, i.e., if $x^{-1}$ exists,
then, for $n \geq 0$, I will define
\[
x^{-n} = \left( x^{-1} \right)^n
\]

\begin{prop}
  Let $x \in \Z$.
  Then $[n]x = n \cdot x$.
\end{prop}

Note that $nx$ has two meanings: $nx$ can be the multiplication of $n$
and $x$ and it can also be $x + \cdots + x$ with $n$ number of $x$.
Of course you would expect them to be the same.
For now define
\[
  [n]x =
  \begin{cases}
    0       &\text{if } n = 0 \\
    [n-1]x + x &\text{if } n > 0 \\
  \end{cases}
\]
and if $n$ is negative, we define
\[
[n]x = -([-n]x)
\]

\newpage
\section{Divisibility}
\begin{defn}
  Let $m, n \in \Z$.
  Then we say that $m$ \defterm{divides} $n$, and we write $m \mid n$, if
  there is some $x \in \Z$ such that $mx = n$, i.e.,
  \[
    \exists x \in \Z \cdot [mx = n]
  \]
\end{defn}


\begin{prop}
  Let $a \in \Z$.
  \begin{enumerate}[nosep,label=\textnormal{(\alph*)}]
  \item Then $1 \mid a$.
  \item Then $a \mid 0$.
  \item (Reflexive) $a \mid a$.
  \item If $a \mid b$ and $b \mid a$, then $a = \pm b$.  
  \item (Transitive) If $a \mid b$ and $b \mid c$, then $a \mid c$.
  \item If $a \mid b$, then $a \mid bc$.
  \item If $a \mid b$, $a \mid c$, then $a \mid b + c$.
  \item (Linearity)
    If $a \mid b$, $a \mid c$, then $a \mid bx + cy$ for $x,y \in \Z$.  
  \end{enumerate}
\end{prop}
\proof
TODO

\newpage
\section{Congruences}

\begin{defn}
  Let $a, b \in \Z$ and $N \in \Z$ with $N > 0$.
  Then $a$ is congruent to $b$ mod $N$ and we write
  \[
  a \equiv b \pmod{N}
  \]
  if $N \mid a - b$.
\end{defn}


\begin{prop}
  Let $a,b,c,a',b' \in \Z$.
  \begin{enumerate}[nosep,label=\textnormal{(\alph*)}]
  \item (Reflexivity) $a \equiv a \pmod{N}$
  \item (Symmetry) If $a \equiv b \pmod{N}$, then $b \equiv a \pmod{N}$
  \item (Transitivity)
    If $a \equiv b, b \equiv c \pmod{N}$, then
    $a \equiv c \pmod{N}$
  \item
    If
    $a \equiv b$,
    $a' \equiv b' \pmod{N}$,
    then
    $a + a' \equiv b + b' \pmod{N}$.
  \item
    If
    $a \equiv b$,
    $a' \equiv b' \pmod{N}$,
    then
    $a a' \equiv b b' \pmod{N}$.
  \end{enumerate}
\end{prop}
\proof
TODO

\begin{prop}
  Let $a, N \in \Z$ with $N > 0$.
  Let $q, r \in \Z$ such that
  \[
  a = Nq + r, \,\,\, 0 \leq r < N
  \]
  Then $a \equiv r \pmod{N}$.
\end{prop}
\proof
TODO

\begin{ex}
  Show that the cancellation law for $\Z$ does not translate to $\Z$ mod $N$.
  In other words, find $N, a, b, c$ such that $c \not\equiv 0 \pmod{N}$ and
  \[
  ac \equiv bc \pmod{N}, \,\,\, a \not\equiv b \pmod{N}
  \]
\end{ex}

\newpage\section{Euclidean algorithm -- GCD}

Now let me use the Euclidean property to compute the gcd of two integers.

Let's use the division algorithm on $20$ and $6$.
\[
20 = 6 \cdot 3 + 2, \,\,\,\,\, 0 \leq 2 < 6
\]
Suppose I want to compute $\gcd(20, 6)$.
Of course the example is small enough that we know that it is $2$.
But notice something about this:
\[
20 = 6 \cdot 3 + 2, \,\,\,\,\, 0 \leq 2 < 6
\]
If $d$ is a divisor of $20$ and $6$, then it must also divide
$2$.
Therefore $\gcd(20, 6)$ must divide $2$.
The converse might not be true.
In general, we have this crucial
bridge between Euclidean property and 
common divisors:

\begin{lem}
  \textnormal{(\defone{GCD Lemma})}
  If $a,b,q,r \in \Z$ such that  
  \[
  a = bq + r
  \]
  then
  \[
  \{ d \mid \text{d is a common divisor of $a, b$} \}
  =
  \{ d \mid \text{d is a common divisor of $b, r$} \}
  \]
  Hence
  \[
  \gcd(a,b) = \gcd(b, r)
  \]
\end{lem}
\proof
TODO


In particular, 
given $a, b \in \Z$ where $a > b > 0$.
By the Euclidean property of $\Z$, there exist $q,r \in \Z$ such that
\[
a = bq + r, \,\,\, 0 \leq r < b
\]
Hence
\[
\gcd(a,b) = \gcd(b, r)
\]

Note that in the above, I only require $a = bq + r$.
For instance for to $\gcd(120, 15)$, I can use
$120 = 1\cdot 15 + (120 - 15)$, i.e., $a = 120, b = 15, q = 1, r = 120-15$.
Then $\gcd(120, 15) = \gcd(15, 120-15) = \gcd(15, 105)$.

However if I use the division algorithm, then 
$r$ is \lq\lq small'':
\[
0 \leq r < b
\]
So if you want to compute $\gcd(a,b)$, make sure $a \geq b$ (otherwise
swap them).
Then $\gcd(a, b) = \gcd(b, r)$ and you would have $a \geq b > r$.
So instead of computing $\gcd(a, b)$, you are better off
computing $\gcd(b, r)$.

But like I said, we do not need the $q$ and $r$ to be the quotient
and remainder.
For instance suppose I want to compute the GCD of 514 and 24.
\[
514 = 24 \cdot 1 + (514 - 24)
\]
Then 
\[
\gcd(514, 24) = \gcd(24, 514 - 24)
\]
which gives us
\[
\gcd(514, 24) = \gcd(24, 490)
\]

Note that $\gcd(0, n) = n$ for any positive integer $n$.
I'll let you think about that one.
(Remember what I said before: 0 is in some sense a big number,
like a black hole. Because every positive number divides $0$.)

Of course this gives rise to the following algorithm
\begin{Verbatim}[frame=single,fontsize=\footnotesize]
ALGORITHM: GCD
Inputs: a, b
Output: gcd(a, b)

if b > a:
    swap a, b

if b == 0:
    return a
else:
    return GCD(a - b, b) 
\end{Verbatim}

This only subtracts one copy of $b$ from $a$.
Suppose we can compute
\[
a = bq + r, \,\,\,\,\, 0 \leq r < b
\]
Then
\[
\gcd(a, b) = \gcd(b, r) 
\]
Of course $r$ is the remainder when $a$ is divided by $b$.
Using this we rewrite the above code to get
the
\defterm{Euclidean Algorithm}\index{Euclidean Algorithm}\tinysidebar{Euclidean Algorithm}:
\begin{Verbatim}[frame=single,fontsize=\footnotesize]
ALGORITHM: GCD (Euclidean algorithm)
Inputs: a, b
Output: gcd(a, b)

if b > a:
    # To make sure that for gcd(a,b), a >= b
    swap a, b

if b == 0: 
    return a
else:
    return GCD(b, a % b)
\end{Verbatim}
Note that if \verb!a! $<$ \verb!b!, then
\[
\texttt{GCD(a, b) = GCD(b, a \% b) = GCD(b, a)}
\]
Therefore the swap is not necessary:
\begin{Verbatim}[frame=single,fontsize=\footnotesize]
ALGORITHM: GCD (Euclidean algorithm)
Inputs: a, b
Output: gcd(a, b)

if b == 0: 
    return a
else:
    return GCD(b, a % b)
\end{Verbatim}
In this case, I'm assuming that \verb!a % b! is available.
As an example:
\begin{align*}
  \gcd(514, 24)
  &= \gcd(24, 514 \% 24) = \gcd(24, 10) \\
  &= \gcd(10, 24 \% 10) = \gcd(10, 4) \\
  &= \gcd(10, 10 \% 4) = \gcd(10, 2) \\
  &= \gcd(2, 10 \% 2) = \gcd(2, 0) \\
  &= 2
\end{align*}

The above can also be done in a loop:
\begin{Verbatim}[frame=single,fontsize=\footnotesize]
ALGORITHM: GCD (Euclidean algorithm)
Inputs: a, b
Output: gcd(a, b)

while 1:
    if b == 0: 
        return a
    else:
        a, b = b, a % b
\end{Verbatim}

\begin{ex}
  Compute the following using the Euclidean Algorithm explicitly.
  \begin{enumerate}[nosep,label=\textnormal{(\alph*)}]
  \item $\gcd(10, 1)$
  \item $\gcd(10, 10)$
  \item $\gcd(107, 5)$
  \item $\gcd(107, 26)$
  \item $\gcd(84, 333)$
  \end{enumerate}
\end{ex}

\begin{ex}
  Compute the following.
  You should go as far as you can.
  In other words, either you can a fixed integer (such as $1$) or
  derive the $\gcd(\alpha, \beta)$ where $\alpha, \beta$ are
  as simple as possible.
  For instance, to simplify $\gcd(3 + 2a, a)$, since $3 + 2a = 2 \cdot a + 3$,
  we have
  \[
  \gcd(3 + 2a, a) = \gcd(a, 3)
  \]
  In the following $a, b, x, n \in \Z$ are positive integers.
  \begin{enumerate}[nosep,label=\textnormal{(\alph*)}]
  \item $\gcd(ab, b)$
  \item $\gcd(a, a + 1)$
  \item $\gcd(ab + a, b)$ where $0 < a < b$
  \item $\gcd(a(a+1) + a, (a+1))$ where $0 < a < b$
  \item $\gcd(1 + x + \cdots + x^n, x)$
  \item $\gcd(F_{10}, F_{11})$ where $F_n$ is the $n$--th Fibonacci number.
    (Recall: $F_0 = 1, F_1 = 1, F_{n + 2} = F_{n + 1} + F_n$ for $n \geq 0$.)
  \end{enumerate}
\end{ex}

Despite the fact that the Euclidean algorithm is one of the fastest
algorithm to compute the GCD of two numbers and has been discovered
by
\href{https://en.wikipedia.org/wiki/Euclid}{Euclid} a long time ago (BC 300),
the actual runtime was not known until
\href{https://en.wikipedia.org/wiki/Gabriel_Lam%C3%A9}{Lam\'e}
proved in 1844 that the number of steps to compute
$\gcd(a, b)$ using the Euclidean algorithm
is $\leq 5$ times the number of digits (in base 10 notation)
of $\min(a, b)$.
For instance for the example above
of $\gcd(514, 24)$, the number of digits of $\min(514,24)$ is $2$.
Lam\'e theorem says that the number of steps made by the
Euclidean algorithm in the computation of $\gcd(514, 24)$ is
at most $5 \times 2 = 10$.
The actual number of steps in the earlier computation
\begin{align*}
  \gcd(514, 24)
  &= \gcd(24, 514 \% 24) = \gcd(24, 10) \\
  &= \gcd(10, 24 \% 10) = \gcd(10, 4) \\
  &= \gcd(10, 10 \% 4) = \gcd(10, 2) \\
  &= \gcd(2, 10 \% 2) = \gcd(2, 0) \\
  &= 2
\end{align*}
is 4 (not counting the base case step), i.e.,
\[
  \gcd(514, 24)
 = \gcd(24, 10) \\
 = \gcd(10, 4) \\
 = \gcd(10, 2) \\
 = \gcd(2, 0) \\
\]
Lam\'e's work is generally considered the beginning of
computational complexity theory, which is the study of
resources needed (time or space) to execute an algorithm.
Another fascinating fact about Lam\'e's theorem is that historically
the above proof is the first ever \lq\lq use" of the Fibonacci sequence.

\begin{thm} \textnormal{(Lam\'e 1844)}
  Let $a > b > 0$ be integers.
  If the GCD computation of $a,b$ 
  using Euclidean algorithm results in $n$ steps:
  \[
  \gcd(a_{n+1}, b_{n+1})
  = \gcd(a_{n}, b_{n})
  = \cdots
  = \gcd(a_1, b_1), \,\,\, b_1 = 0
  \]
  where $(a_{n+1}, b_{n+1}) = (a, b)$, and $a_i > b_i$, then
  \begin{enumerate}[nosep,label=\textnormal{(\alph*)}]
  \item  $a \geq F_{n+2}$ and
    $b \geq F_{n+1}$, where $F_n$ are the Fibonacci numbers
    ($F_1 = 1, F_2 = 1, F_3 = 2, F_4 = 3, F_5 = 5$, etc.
    Note that the index starts with $1$.)
  \item $n$ is at most 5 times the number of digits in $b$.
  \end{enumerate}
\end{thm}
\proof
TODO
  
\begin{prop}
  The number of digits of a positive integer $b$ is
  \[
  \floor{\log_{10} b + 1}
  \]
\end{prop}
\proof
TODO

On analyzing the proof, the above is in fact true for any base $> 1$.
In other words the number of base--$B$ symbols to represent $b$ is
\[
\floor{\log_{B} b + 1}
\]
where $B > 1$.
For instance the number of bits needed to represent $b$ is
\[
\floor{\log_{2} b + 1}
\]
For instance, $b = 9_{10} = 1001_2$ which has 4 bits and
\[
\floor{\log_{2} 9 + 1}
= \floor{3.1699... + 1}
= \floor{4.1699...}
= 4
\]

\begin{prop}
  Let positive integer $b$ be written in base $B$ where $B > 1$ is an integer.
  Then the number of base-$B$ symbols used to represent $b$ is
  \[
  \floor{\log_{B} b + 1}
  \]
\end{prop}

\begin{ex}
  Leetcode 650.
  \\
  \url{https://leetcode.com/problems/2-keys-keyboard/}
  \\
  There is only one character \verb!'A'! on the screen of a notepad.
  You can perform one of two operations on this notepad for each step:
  \begin{enumerate}[nosep]
    \li Copy All:
    You can copy all the characters present on the screen (a partial copy is
    not allowed).
    \li Paste: You can paste the characters which are copied last time.
  \end{enumerate}
  Given an integer \verb!n!, return the minimum number of operations to get the character \verb!'A'! exactly \verb!n! times on the screen.
\end{ex}


\begin{ex}
  Leetcode 2447.
  \\
  \url{https://leetcode.com/problems/number-of-subarrays-with-gcd-equal-to-k}
  \\
  Given an integer array \verb!nums! and an integer \verb!k!,
  return the number of subarrays of \verb!nums! where the greatest common
  divisor
  of the subarray's elements is \verb!k!.
  A subarray is a contiguous non-empty sequence of elements within an array.
  The greatest common divisor of an array is the largest integer that evenly
  divides all the array elements.
\end{ex}

\begin{ex}
  Leetcode 1998
  \\
  \url{https://leetcode.com/problems/gcd-sort-of-an-array/}
  \\
  You are given an integer array \verb!nums!,
  and you can perform the following operation any number of times on
  \verb!nums!:
  \begin{enumerate}[nosep]
    \li Swap the positions of two elements \verb!nums[i]! and \verb!nums[j]! if
    \verb!gcd(nums[i], nums[j]) > 1! where \verb!gcd(nums[i], nums[j])! is the
    greatest common divisor of \verb!nums[i]! and \verb!nums[j]!.
  \end{enumerate}
  Return true if it is possible to sort \verb!nums! in non-decreasing order
  using the above swap method, or false otherwise.
\end{ex}

\newpage\section{Extended Euclidean algorithm: GCD as linear combination}

Here's another super important fact:

\begin{thm}
  \textnormal{(\defterm{Extended Euclidean Algorithm}\index{Extended Euclidean Algorithm}\tinysidebar{Extended Euclidean Algorithm})}
If $a$ and $b$ be integers which are not both zero,
then there are integers $x,y$ such that
\[
\gcd(a,b) = ax + by
\]
\end{thm}

The $x,y$ in the above theorem are called
\defterm{B\'ezout's coefficients}\index{B\'ezout's coefficients}\tinysidebar{B\'ezout's coefficients}
of $a,b$.
They are not unique.

\begin{ex}
  Prove that $a \neq 0$, then
  there are many possible $x, y$ such that
  $ax + by = \gcd(a,b)$.
  \qed
\end{ex}
% ax + by = ax + by + ab - ab = a(x + b) + b(y - a).
% ax + by = ax + by + nab - nab = a(x + nb) + b(y - na).


First let me prove that there are $x,y$ such that
\[
ax + by = \gcd(a, b)
\]
The theorem does not give you the algorithm.
Then I'll do a computational example
that compute the gcd of $a,b$ as a linear combination of $a$ and $b$.
The example actually contains the idea behind the algorithm
to compute the B\'ezout's coefficients.
The algorithm is called
the Extended Euclidean Algorithm.

\proof
For convenience, let
me write $(a, b)$ as the set of linear combinations of $a$ and $b$, i.e.,
\[
(a, b) = \{ax + by \mid x,y \in \Z \}
\]
We will also write $(g)$ for the linear combination of $g$, i.e.,
\[
(g) = \{gx \mid x \in \Z \}
\]
(Such linear combinations are called ideals. They are extremely
important in of themselves. Historically, they were created
to study Fermat's last theorem. Since then they are crucial in the
the study of ring theory.)

The proof proceeds in two steps:
\begin{enumerate}[nosep]
\item Given $a,b$ not both zero, there is some $g$ such that
  $(a,b) = (g)$. If $a,b$ are not both zero, $g$ can be chosen to be
  $>0$.
\item The $g$ in the above is in fact $\gcd(a,b)$. 
\end{enumerate}

Now let's prove step 1, i.e., given $a,b$, 
there is some $g$ such that
\[
(a, b) = (g)
\]
First of all if $b = 0$, then by definition
\[
(a, 0) = (a)
\]
and we're done.
Next, we now assume $b \neq 0$.
Then $|b| > 0$.
The set
\[
X = \{ax + by \mid x,y \in \Z \text{ and } ax + by > 0 \} \subseteq \N
\]
is nonempty since it contains
$0\cdot a + 1\cdot |b|$.
By the well-ordering principle of $\N$, $X$ has a minimum element, say $g$.
We will now show that $(a, b) = (g)$.

Since $g$ is a minimum element of $X$, $g$ is in $X$.
Therefore $g = ax + by$.
Hence $gz = a(xz) + b(yz) \in (a, b)$ for all $z \in \Z$.
This implies that $(g) \subseteq (a, b)$.

Now we will prove that $(a, b) \subseteq (g)$.
Let $c \in (a, b)$, i.e., $c = ax + by$ for some $x,y \in \Z$.
Therefore by the Euclidean property of $\Z$, there exists $q,r \in \Z$ such that
\[
c = gq + r, \,\,\, 0 \leq r < g
\]
(Look at the definition of $X$ again. $X$ is a subset of $\N$ so that
$g \geq 1$).
If $r \neq 0$, then
\[
r = c - gq
\]
Note that $c = ax + by$ by our assumption.
We have already shown that $(g) \subseteq (a, b)$, i.e., $g = ax' + by'$.
Therefore, altogether we have
\[
r = c - gq = ax + by - (ax' + b'y)q = a(x-x'q) + b(y-y'q)
\]
Hence $r \in X$.
But $0 \leq r < d$ implies that
\[
r = a(x-x'q) + b(y-b'q)
\]
is an element of $X$ which is less than $g$ which contradicts the
minimality of $g$.
Hence $r = 0$ and we have
\[
c = gq + r = gq \in (g)
\]
We have shown that $(a,b) \subseteq (g)$.

Altogether, we have shown $(a,b) = (g)$.
Step 1 is now completed.

For step 2, we will show that $g$ is the gcd of $a$ and $b$.
Since $(a,b) = (g)$, we have
\[
a \in (a,b) = (g)
\]
i.e., $a = xg$ which means $g$ divides $a$.
Likewise $g$ divides $b$.
Hence $g$ is a common divisor of $a$ and $b$.
Since $(g) \subseteq (a, b)$,
$g = ax_0 + by_0$.
Suppose $d$ is any divisor of $a$ and $b$.
Then $d \mid ax_0 + by_0$ by the linearity of divisibility.
Hence $d \mid g$.
Therefore $g$ is the largest common divisor of $a$ and $b$,
i.e., $g = \gcd(a,b)$.
\qed

The above does not give you an algorithm to compute the
$x$ and $y$.
First let me do an example to show you that
it's possible to compute $\gcd(a,b)$ as a linear combination of $a$ and $b$.
Then I'll give you the algorithm.

Recall that we have computed $\gcd(514, 24) = 2$.
Extended Euclidean Algorithm says that it's possible to find $x$ and $y$
such that
\[
2 = \gcd(514, 24) = 514x + 24y
\]
How do we compute the $x$ and $y$?
Just like the gcd computation (the Euclidean Algorithm),
the $x,y$ are computed using the Euclidean property.
First we have
\begin{align*}
514 = 21 \cdot 24 + 10
\end{align*}
This implies that 
\[
514 \cdot 1 + 24 \cdot (-21) = 10
\]
Now it would be nice if the pesky $10$ goes away and is replaced by
$2$.
How would we do that?
Well look at $24$ and $10$ now.
We have
\[
24 = 2 \cdot 10 + 4
\]
again by Euclidean algorithm.
Multiplying the equation
\[
514 \cdot 1 + 24 \cdot (-21) = 10
\]
throughout by 2 gives us
\[
514 \cdot 2 + 24 \cdot (-42) = 2 \cdot 10
\]
The previous equation
\[
24 = 2 \cdot 10 + 4
\]
say that $2 \cdot 10$ can be replaced by $24 - 4$.
This means that
\[
514 \cdot 2 + 24 \cdot (-42) = 24 - 4
\]
Hmmm ... this says that we have now
\[
514 \cdot 2 + 24 \cdot (-43) = -4
\]
or
\[
514 \cdot (-2) + 24 \cdot 43 = 4
\]
What about 4?
Well, if we look at $10$ and $4$ just like what we did with $24$ and $10$
we would get
\[
10 = 2 \cdot 4 + 2
\]
and the remainder $2$ gives us the GCD!!!
Rearranging it a bit we have
\[
1 \cdot 10 + (-2) \cdot 4 = 2
\]
i.e. 2 is a linear combination of 10 and 4.
But earlier we say that 4 is a linear combination of 514 and 24 ...
\[
514 \cdot (-2) + 24 \cdot 43 = 4
\]
and even earlier we saw that 10 is also a linear combination of 514 and 24 ...
\[
514 \cdot 1 + 24 \cdot (-21) = 10
\]
Surely if we substitute all these values into the equation
\[
1 \cdot 10 + (-2) \cdot 4 = 2
\]
we would get 2 as a linear combination of 514 and 24.
Let's do it ...
\begin{align*}
2 
&= 1 \cdot 10 + (-2) \cdot 4 \\
&= 1 \cdot (514 \cdot 1 + 24 \cdot (-21)) + (-2) (514 \cdot (-2) + 24 \cdot 43) \\
&= 514 \cdot 1 + 24 \cdot (-21) + 514 \cdot 4 + 24 \cdot (-86) \\
&= 514 \cdot 5 + 24 \cdot (-107)
\end{align*}
V\'oila!

\newpage
\begin{ex}
  Using the above idea,
  compute the $\gcd$ and B\'ezout's coefficients of 210 and 78, i.e.,
  compute $x$ and $y$ such that $210x + 78y = \gcd(210, 78)$.
\end{ex}

\newpage
\begin{ex}
Analyze the above and design an algorithm 
so that when given $a$ and $b$, the algorithm computes
$x$ and $y$ such that $ax + by = \gcd(a,b)$.
\end{ex}

\newpage
To help you analyze the above computation,
let me organize our computations a little.
If we can make the process systematic, then there is hope that we can 
make the idea work for all $a$ and $b$, i.e., then we would have an
algorithm and hence can program it and compute it's runtime performance.

We know for sure that we have to continually use
Euclidean property on pairs of numbers.
So here we go:
\begin{align*}
514 &= 21 \cdot 24 + 10 \\
24  &= 2 \cdot 10 + 4 \\
10  &= 2 \cdot 4 + 2 \\
4   &= 2 \cdot 2 + 0
\end{align*}
Note that this corresponds to the $\gcd$ computation
\begin{align*}
\gcd(514, 24) 
&= \gcd(24, 514 - 21 \cdot 24) = \gcd(24, 10) \\
&= \gcd(10, 24 - 2 \cdot 10) = \gcd(10, 4) \\
&= \gcd(4, 10 - 2 \cdot 4) = \gcd(4, 2)  \\
&= \gcd(2, 4 - 2\cdot 2) = \gcd(2, 0) \\
&= 2
\end{align*}
So in the computation
\begin{align*}
514 &= 21 \cdot 24 + 10 \\
24  &= 2 \cdot 10 + 4 \\
10  &= 2 \cdot 4 + 2 \\
4   &= 2 \cdot 2 + 0
\end{align*}
if the remainder is $0$ (see the last line), 
then the previous line's remainder must be the gcd.

Let's look at our computation of the gcd of $514$ and $24$:
\begin{align*}
514 &= 21 \cdot 24 + 10 \\
24  &= 2 \cdot 10 + 4 \\
10  &= 2 \cdot 4 + 2 \\
4   &= 2 \cdot 2 + 0
\end{align*}
Recall that the above computation means that the gcd is 2.
Note only that through backward substitution, we can rewrite
$2$ as a linear combination of $514$ and $24$.

Let's try to do this in a more organized way.
So here's our facts again:
\begin{align*}
514 &= 21 \cdot 24 + 10 \\
24  &= 2 \cdot 10 + 4 \\
10  &= 2 \cdot 4 + 2 
\end{align*}
Let me put the remainders on one side:
\begin{align*}
10 &= 514 - 21 \cdot 24 \tag{1} \\
4  &= 24  - 2 \cdot 10  \tag{2} \\
2  &= 10  - 2 \cdot 4   \tag{3} \\
\end{align*}
Note that (1) tells you that
$10$ is a linear combination of $514,24$.
(2) tells you that $4$ is a linear combination of $24,10$.
If we substitute (1) into (2), $4$ will become a linear
combination of $514,24$.
(3) says that $2$ is a linear combination of $10,4$.
But $10$ is a linear combination of $514,24$
and $4$ is a linear combination of $514,24$.
Hence $2$ is also a linear combination of $514,24$.
See it?

OK. Let's do it.
From
\begin{align*}
10 &= 514 - 21 \cdot 24 \tag{1} \\
4  &= 24  - 2 \cdot 10  \tag{2} \\
2  &= 10  - 2 \cdot 4   \tag{3} \\
\end{align*}
if we substitute (1) into (2) and (3) (i.e., rewrite $10$ as a linear
combination of $514,24$):
\begin{align*}
10 &= 514 - 21 \cdot 24 \tag{1} \\
4  &= 24  - 2 \cdot (514 - 21 \cdot 24)  \tag{2} \\
2  &= (514 - 21 \cdot 24)  - 2 \cdot 4   \tag{3} 
\end{align*}
Collecting the multiples of 514 and 24:
\begin{align*}
10 &= 514 - 21 \cdot 24 \tag{1} \\
4  &= (-2) \cdot 514 + (1 + (-2)(-21)) \cdot 24  \tag{2'} \\
2  &= (1) \cdot 514 + (-21) \cdot 24  - 2 \cdot 4   \tag{3'} 
\end{align*}
and simplifying:
\begin{align*}
10 &= 514 - 21 \cdot 24 \tag{1} \\
4  &= (-2) \cdot 514 + (43) \cdot 24  \tag{2'} \\
2  &= (1) \cdot 514 + (-21) \cdot 24  - 2 \cdot 4   \tag{3'} 
\end{align*}
Substituting (2$'$) into (3$'$):
\begin{align*}
10 &= 514 - 21 \cdot 24 \tag{1} \\
4  &= (-2) \cdot 514 + (43) \cdot 24  \tag{2'} \\
2  &= (1) \cdot 514 + (-21) \cdot 24  - 2 \cdot ((-2) \cdot 514 + (43) \cdot 24)  \tag{3'} 
\end{align*}
Tidying up:
\begin{align*}
10 &= 514 - 21 \cdot 24 \tag{1} \\
4  &= (-2) \cdot 514 + (43) \cdot 24  \tag{2'} \\
2  &= (1 - 2(-2)) \cdot 514 + (-21 - 2(43)) \cdot 24 \tag{3''} 
\end{align*}
Simplifying:
\begin{align*}
10 &= 514 - 21 \cdot 24 \tag{1} \\
4  &= (-2) \cdot 514 + (43) \cdot 24  \tag{2'} \\
2  &= (5) \cdot 514 + (-107) \cdot 24 \tag{3''}
\end{align*}

(It's a good idea to check after each substitution that
the equalities still hold. We all make mistakes, right?)

OK.
That's great.
It looks more organized now.
So much so that you can now easily write a program to compute
the above.


Now let's look at the general case.
Suppose instead of $514$ and $24$, we write $a$ and $b$.
The computation will look like this:
\begin{align*}
a   &= q_1 \cdot b   + r_1 \\
b   &= q_2 \cdot r_1 + r_2 \\
r_1 &= q_3 \cdot r_2 + r_3 \\
r_2 &= q_4 \cdot r_3 + 0
\end{align*}
To make things even more regular and uniform, let me rewrite it this way:
\begin{align*}
r_0 &= q_1 \cdot r_1 + r_2 \\
r_1 &= q_2 \cdot r_2 + r_3 \\
r_2 &= q_3 \cdot r_3 + r_4 \\
r_3 &= q_4 \cdot r_4 + 0
\end{align*}
A lot nicer, right?
Let me write it this way with the remainder term on the lefts:
\begin{align*}
r_2 &= (1) \cdot r_0 + (-q_1) \cdot r_1 \\
r_3 &= (1) \cdot r_1 + (-q_2) \cdot r_2 \\
r_4 &= (1) \cdot r_2 + (-q_3) \cdot r_3 
\end{align*}
(Remember that $r_4$ is the gcd ... $r_0 = 514, r_1 = 24$ ... right?)
Organized this way, we have the gcd on one side of the equation.
Now if we substitute the first equation into the second we get
\begin{align*}
r_2 &= (1) \cdot r_0 + (-q_1) \cdot r_1 \text{ ... USED }\\
r_3 &= (1) \cdot r_1 + (-q_2) \cdot ((1) \cdot r_0 + (-q_1) \cdot r_1) \\
r_4 &= (1) \cdot r_2 + (-q_3) \cdot r_3 
\end{align*}
i.e.,
\begin{align*}
r_2 &= (1) \cdot r_0 + (-q_1) \cdot r_1 \text{ ... USED }\\
r_3 &= (-q_2) \cdot r_0 + (1 + q_1q_2) \cdot r_1 \\
r_4 &= (1) \cdot r_2 + (-q_3) \cdot r_3  
\end{align*}
Note that we cannot throw away the first equation yet.
We need to keep $r_2$ around since it appears in the third equation!
So when can we throw the first equation away?
Look at the general case.
Suppose we have
\begin{align*}
r_2 &= (1) \cdot r_0 + (-q_1) \cdot r_1  \\
r_3 &= (1) \cdot r_1 + (-q_2) \cdot r_2  \\
r_4 &= (1) \cdot r_2 + (-q_3) \cdot r_3  \\
r_5 &= (1) \cdot r_3 + (-q_4) \cdot r_4  \\
r_6 &= (1) \cdot r_4 + (-q_5) \cdot r_5  \\
...
\end{align*}
Aha! $r_2$ is used only in the next \textit{two} equations.

Suppose we are at equation 3:
\begin{align*}
r_3 &= c_1 \cdot r_0 + d_1 \cdot r_1  \\
r_4 &= c_2 \cdot r_0 + d_2 \cdot r_1 
\end{align*}
We have to compute the next equation:
This requires $r_3, r_4$.
Then we have
\[
r_5 = (1) \cdot r_3 + (-q_4) \cdot r_4
\]
where
\[
q_4 = \floor{r_3/r_4}, \,\,\,\,\, r_5 = r_3 - q_4r_4
\]
Altogether we have
\begin{align*}
r_3 &= c_1 \cdot r_0 + d_1 \cdot r_1     \\
r_4 &= c_2 \cdot r_0 + d_2 \cdot r_1     \\
r_5 &= (1) \cdot r_3 + (-q_4) \cdot r_4 
\end{align*}
The last equation becomes
\begin{align*}
r_5 &= c_1 \cdot r_0 + d_1 \cdot r_1  
+ 
(-q_4) \cdot (c_2 \cdot r_0 + d_2 \cdot r_1)
\end{align*}
i.e.
\begin{align*}
r_5 &= (c_1 - q_4 c_2) \cdot r_0 + (d_1 - q_4 d_2) \cdot r_1 
\end{align*}

Let me repeat that in a slightly more general context.
If we have
\begin{align*}
r_3 &= c_1 \cdot r_0 + d_1 \cdot r_1  \\
r_4 &= c_2 \cdot r_0 + d_2 \cdot r_1 
\end{align*}
then we get (throwing away the first equation):
\begin{align*}
r_4 &=c_2 \cdot r_0 + d_2 \cdot r_1 \\
r_5 &= (c_1 - q_4 c_2) \cdot r_0 + (d_1 - q_4 d_2) \cdot r_1 
\end{align*}
To put it in terms of numbers instead of equations this is what we get:
If we have
\[
c_1, d_1, c_2, d_2, r_3, r_4
\]
then we get
\[
c_2, d_2, 
c_1 - \floor{r_3/r_4} c_2, 
d_1 - \floor{r_3/r_4}d_2, 
r_4, r_3 - \floor{r_3/r_4}r_4
\]
In general, if we have
\[
c, d, c', d', r, r'
\]
then we get
\[
c', d', 
c - \floor{r/r'} c', d - \floor{r/r'}d', 
r', r - \floor{r/r'}r'
\]

Of course since we start off with $r_0, r_1$ (i.e. what we call $a$ and $b$
above), we have
\begin{align*}
r_0 &= 1 \cdot r_0 + 0 \cdot r_1 \\
r_1 &= 0 \cdot r_0 + 1 \cdot r_1 \\
\end{align*}
i.e., you would start off with
\[
c=1, d=0, c'=0, d'=1, r=r_0, r'=r_1
\]
Let's check this algorithm with our $r_0 = 514, r_1 = 24$.

STEP 1: The initial numbers are
\[
c=1, d=0, c'=0, d'=1, r=514, r'=24
\]
Again this corresponds to
\begin{align*}
r_3 &= 1 \cdot 514 + 0 \cdot 24  \\
r_4 &= 0 \cdot 514 + 1 \cdot 24 \\
\end{align*}

STEP 2: 
The new numbers (6 of them) are
\begin{align*}
 c' &=0 \\
 d' &=1 \\
 c - \floor{r/r'} c' &= 1 - \floor{514/24} 0 = 1\\
 d - \floor{r/r'}d' &= 0 - \floor{514/24}1 = 0 - 21 = -21 \\
 r' &= 24 \\ 
 r - \floor{r/r'}r' &= 514 - \floor{514/24}24 = 514 - 504 = 10
\end{align*}
So the new numbers (we reset the variables in the algorithm):
\[
c=0, d=1, c'=1, d'=-21, r=24, r'= 10
\]
These corresponds to the data on the second and third line of the following:
\begin{align*}
514 &= 1 \cdot 514 + 0 \cdot 24      \\
24  &= 0 \cdot 514 + 1 \cdot 24      \\
10  &= 1 \cdot 514 + (-21) \cdot 24 
\end{align*}

STEP 3: From the 6 numbers from STEP 2 we get
\begin{align*}
 c' &= 1 \\
 d' &= -21 \\
 c - \floor{r/r'} c' &= 0 - \floor{24/10} 1 = -2 \\
 d - \floor{r/r'}d'  &= 1 - \floor{24/10} (-21) = 1 + 42 = 43 \\
 r' &= 10 \\ 
 r - \floor{r/r'}r' &= 24 - \floor{24/10}10 = 24 - 20 = 4
\end{align*}
So the new numbers (we reset the variables in the algorithm):
\[
c=1, d=-21, c'=-2, d'= 43, r=10, r'= 4
\]
These corresponds to the data on the third and fourth line of the following:
\begin{align*}
514 &= 1 \cdot 514 + 0 \cdot 24     \\
24  &= 0 \cdot 514 + 1 \cdot 24     \\
10  &= 1 \cdot 514 + (-21) \cdot 24 \\
4   &= (-2) \cdot 514 + 43 \cdot 24 
\end{align*}

STEP 4: From the 6 numbers from STEP 3 we get
\begin{align*}
 c' &= -2 \\
 d' &= 43 \\
 c - \floor{r/r'} c' &= 1 - \floor{10/4} (-2) = 1 + 4 = 5 \\
 d - \floor{r/r'}d'  &= -21 - \floor{10/4} (43) = -21 - 86 = -107 \\
 r' &= 4 \\ 
 r - \floor{r/r'}r' &= 10 - \floor{10/4} 4 = 10 - 8 = 2
\end{align*}
So the new numbers (we reset the variables in the algorithm):
\[
c = -2, d = 43, c' = 5, d'= -107, r=4, r'= 2
\]
These corresponds to the data on the fourth and fifth line of the following:
\begin{align*}
514 &= 1 \cdot 514 + 0 \cdot 24      \\
24  &= 0 \cdot 514 + 1 \cdot 24      \\
10  &= 1 \cdot 514 + (-21) \cdot 24  \\
4   &= (-2) \cdot 514 + 43 \cdot 24  \\
2   &= 5 \cdot 514 + (-107) \cdot 24 
\end{align*}

STEP 5: From the 6 numbers from STEP 4 we get
\begin{align*}
 c' &= 5 \\
 d' &= -107 \\
 c - \floor{r/r'} c' &= -2 - \floor{4/2} 5 = -2 - 10 = -12 \\
 d - \floor{r/r'} d'  &= 43 - \floor{4/2} (-107) = 43 + 214 = 257 \\
 r' &= 2 \\ 
 r - \floor{r/r'}r' &= 4 - \floor{4/2} 2 = 4 - 4 = 0
\end{align*}
So the new numbers (we reset the variables in the algorithm):
\[
c = 5, d = -107, c' = -12, d'= 257, r=2, r'= 0
\]
These corresponds to the data on the fifth and sixth line of the following:
\begin{align*}
514 &= 1 \cdot 514 + 0 \cdot 24       \\
24  &= 0 \cdot 514 + 1 \cdot 24       \\
10  &= 1 \cdot 514 + (-21) \cdot 24   \\
4   &= (-2) \cdot 514 + 43 \cdot 24   \\
2   &= 5 \cdot 514 + (-107) \cdot 24  \\
0   &= (-12) \cdot 514 + 257 \cdot 24 
\end{align*}

Of course (as before) at this point, you see that the $r'=0$.
Therefore 
\[
\gcd(514, 24) = 2
\]
and furthermore from $c=5, d=-107$, we get
\[
5 \cdot 514 + (-107) \cdot 24 = \gcd(514, 24) 
\]

Here's a Python implementation with some test code:
\begin{Verbatim}[frame=single, fontsize=\small]
ALGORITHM: EEA
INPUTS: a, b
OUTPUTS: r, c, d where r = gcd(a, b) = c*a + d*b

    a0, b0 = a, b
    d0, d = 0, 1
    c0, c = 1, 0
    q = a0 // b0
    r = a0 - q * b0
    
    while r > 0:
        d, d0 = d0 - q * d, d    
        c, c0 = c0 - q * c, c
    
        a0, b0 = b0, r
        q = a0 // b0
        r = a0 - q * b0

    r = b0
    return r, c, d
\end{Verbatim}
You can pound real hard at the Extended Euclidean Algorithm with this:

By the way, this is somewhat similar to what we call
\textit{tail recursion} (CISS445)
an extremely important technique in functional programming.
All LISP hackers and people working in high performance computing
and compilers swear by it.
You don't see recursion in the above code, but you can 
replace the while-loop with recursion and if
you have a compiler/interpreter that can perform 
true tail recursion, then it would run exactly like the above algorithm.

\begin{ex}
  Leetcode 365:
  \url{https://leetcode.com/problems/water-and-jug-problem/description/}
  and the Die Hard 3 problem
  \url{https://www.math.tamu.edu/~dallen/hollywood/diehard/diehard.htm}.
  You are given two jugs with capacities \verb!jug1Capacity! and
  \verb!jug2Capacity! liters.
  There is an infinite amount of water supply available.
  Determine whether it is possible to measure exactly \verb!targetCapacity!
  liter using these two jugs.

  If \verb!targetCapacity! liters of water are measurable,
  you must have \verb!targetCapacity! liters of water contained within one or
  both buckets by the end.

  Operations allowed:
  \begin{enumerate}[nosep]
    \item Fill any of the jugs with water.
    \item Empty any of the jugs.
    \item Pour water from one jug into another till the other jug is completely
      full, or the first jug itself is empty.
  \end{enumerate}
  
\end{ex}

You'll see that there are times when you're only interested in 
the value of $x$ and not $y$ (or $y$ and not $x$ -- the above is symmetric
about $x$ and $y$).
Do you notice $x$ comes from $c$?
If you analyze the above algorithm, you see immediately that $c$
is computed from $c'$ and $c'$ is computed from $c,c',q$, 
$q$ is computed from $r, r'$, $r$ is computed from $r'$,
and finally (phew!) $r'$ is computed from $r, q, r'$.
Therefore if you're interested in $c$, you don't need to compute $d$ 
or $d'$.
So you can change the EEA to this:

\begin{Verbatim}[frame=single, fontsize=\small]
ALGORTHM: EEA2 (sort of EEA ... without the d, d0)
INPUTS: a, b
OUTPUTS: r, c where r = gcd(a, b) = c*a + d*b for some d

    a0, b0 = a, b
    c0, c = 1, 0
    q = a0 // b0
    r = a0 - q * b0

    while r > 0:   
        c, c0 = c0 - q * c, c
    
        a0, b0 = b0, r
        q = a0 // b0
        r = a0 - q * b0

    r = b0
    return r, c
\end{Verbatim}

Later you'll see why we compute only $c$.
It's not that we have something against $d$.

\newpage
\begin{ex}
  Compute the following gcd and the B\'ezout's coefficients of the
  given numbers by following the Extended Euclidean Algorithm.
  \begin{enumerate}[nosep]
  \item $\gcd(0, 10)$
  \item $\gcd(10, 0)$
  \item $\gcd(10, 1)$
  \item $\gcd(10, 10)$
  \item $\gcd(107, 5)$
  \item $\gcd(107, 26)$
  \item $\gcd(84, 333)$
  \item $\gcd(F_{10}, F_{11})$ where $F_n$ is the $n$--th Fibonacci number.
    (Recall: $F_0 = 0, F_1 = 1, F_{n + 2} = F_{n + 1} + F_n$.)
  \item $\gcd(ab, b)$
  \item $\gcd(a, a + 1)$
  \item $\gcd(ab + a, b)$ where $0 < a < b$. Go as far as you can.
  \item $\gcd(a(a+1) + a, (a+1))$ where $0 < a < b$. Go as far as you can.
  \end{enumerate}
\end{ex}

\begin{ex}
  Prove that if $a \mid c$, $b \mid c$, and $\gcd(a,b) = 1$, then
  $ab \mid c$.
\end{ex}

\begin{ex}
  Prove that if $a \mid c$, $b \mid c$, then
  \[
  \frac{ab}{\gcd(a,b)} \mid c
  \]
\end{ex}


\begin{ex}
  Leetcode 920.
  \url{https://leetcode.com/problems/number-of-music-playlists}
  Your music player contains n different songs. You want to listen to goal songs (not necessarily different) during your trip. To avoid boredom, you will create a playlist so that:

Every song is played at least once.

A song can only be played again only if k other songs have been played.

Given \verb!n!, \verb!goal!, and \verb!k!, return the number of possible playlists that you can create. Since the answer can be very large, return it modulo $109 + 7$.
\end{ex}

