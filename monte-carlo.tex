\section{Monte-Carlo algorithms}

As noted earlier, Fermat and Miller--Rabin prime testing algorithm
are probabilitic algorithm in the sense that if the
return value is \verb!"n is composite"! then you know for sure $n$ is a
composite (i.e. not a prime), 
but if the return value is
\verb!"n is probably a prime"!,
then $n$ can be either be a prime or a
composite.
This is also called a
\defterm{Monte-Carlo algorithm}\tinysidebar{Monte-Carlo algorithm}\index{Monte-Carlo algorithm}
because the result returned is not guaranteed to be true.
It's also called a \defone{false-biased Monte-Carlo algorithm} because the
false case (i.e., \verb!"not a prime"! return value) is always correct whereas the
true case (i.e., is a prime) is only probabilistically true.

There are actually many primality testing algorithms.
Rabin-Miller is only one of many.

In number theory, there is also a theorem called the Prime Number
Theorem that gives an estimate on prime distribution. We won't go
into this because this is extremely technical.

By the way for a long time it was thought that primality test is
``easy".
Note that Miller--Rabin prime testing (and other primality tests)
is easy probabilistically.
A deterministic polynomial runtime
primality test was finally discovered and proven only recently
(2002) by a group of computer scientists, Agrawal, Kayal and Saxena, from India.
The algorithm is now called the
\defterm{AKS primality test}\tinysidebar{AKS primality test}\index{AKS primality test}
algorithm.
If you want some fancy automata notation, the AKS algorithm
says that
\[
  \mathsc{Primes} \in \mathsc{P}
\]
where $\mathsc{Primes}$ denotes the problem (or language) of
testing for primeness and $\mathsc{P}$
denotes the class of polynomial runtime problems.
This $\mathsc{P}$ is the same $\mathsc{P}$ in the famous
\lq\lq$\mathsc{P} = \mathsc{NP}$'' problem.
AKS is however not used in real-world applications because the runtime
is too slow.
There is current ongoing research on improving the performance
of this algorithm.

In real-world applications of Miller-Rabin prime testing algorithm,
to test that a random 2048-bit odd number is a prime,
using $t = 10$ rounds is usually more than enough.

In 2007, a 1039 bit integer was factored with the number field sieve using
400 computers over 11 months.
Nowadays (2019), primes with 2048 bit length is definitely enough -- unless
someone discovered a new factoring algorithm.
In RSA--speak, when you hear \lq\lq RSA 1024-bit key'', it means
the modulus $N = pq$ has 1024 bit.
That means the bit length of each of the two primes is about 512.


\begin{ex}
  Good research project:
  Study the AKS algorithm.
  \qed
\end{ex}
